
%%%% PROCESAR con PdfLaTeX !!!!!


\documentclass[12pt]{book}
\usepackage{geometry}\geometry{top=5cm,bottom=2cm,left=3cm,right=3cm}
\usepackage{amssymb}
\usepackage{amsmath}
\usepackage{graphicx}
\usepackage{txfonts}




\begin{document}
\thispagestyle{empty}

\begin {center}

\includegraphics[scale=.3]{Logo-fcenuba.png}

\medskip
UNIVERSIDAD DE BUENOS AIRES

Facultad de Ciencias Sociales

C\'atedra de Sociolog\'ia


\vspace{3cm}


\textbf{\large LA RADICALIZACION DE LA MODERNIDAD \\ Antony Guidens}

\vspace{2cm}


RESUMEN DE ANTONY GUIDENS LA RADICALIZACION DE LA MODERNIDAD.
\vspace{2cm}

\textbf{C\'atedra de Sociolog\'ia}

\end {center}


\vspace{1.5cm}

\noindent Ciclo b\'asico com\'un
 
\noindent Buenos Aires, 2009

\newpage 
El sociólogo británico nos habla de una modernidad radicalizada. 
\textbf{¿pero en que consiste?}
Cuando hablamos de modernidad lo primero que viene a nuestra mente es innovaciones y tecnología, es decir que lo asociamos con lo nuevo, pero para las ciencias sociales esto es mucho mas profundo, guidens nos dice que la modernidad es un modo de organización social y forma de vida porque tiene un contexto determinado.
\\
\\
Esa modernidad tiene para guidens ciertos rasgos que la diferencian de las sociedades tradicionales, a modo de contexto podemos decir que la modernidad tiene una serie de caracteristicas.
\begin{enumerate}
\item La razón como forma de conocimiento a diferencia de la idea previa en la que la tradición y la religión era de donde provenía el conocimiento valido entonces hay un énfasis en la ciencia y la tecnología, 
\item Los procesos de racionalización en términos de max weber, es la tendencia de hacer las cosas lo mas ordenadas posible predecible y ordenada, por ejemplo los algoritmos de google y los procesos industriales estas pensando en ese sentido moderno
\item En tercer lugar hay una tendencia al individualismo eso significa que en lugar de elegir entre lo que ha sido usual en tu contexto ya sea familia o entorno social, se hace mucho énfasis en tus elecciones personales.

\end{enumerate}
\textbf{DIMENSIÓN INSTITUCIONAL}
\\
Ademas guidens doce que hay una dimensión institucional en la modernidad, es decir que existen ciertas instituciones que han sido propias de la modernidad que son muy cercana a ella, por ejemplo el capitalismo este sistema económico en el que se acumula riqueza atra vez de la producción que esta en manos de aquellos que han acumulado capital suficiente para ello.
Se salta de una pequeña producción de talleres que había antes en la antigüedad a una escala Masiva.
\\
A nivel institución política tenemos el surgimiento del estado nación, es decir antes la organización política era en grandes imperios eso cambia con la modernidad.
\\
Guidenns refuta la idea de otros autores que dicen que la modernidad se termino y estamos hoy en la posmodernidad entendida como una nueva etapa, caracterizada por una nueva actitud por la vida desencantada de las promesas de la modernidad que nos vendió que gracias al uso de la razón ibamos a tener una sociedad mas justa pero en cambio lo que hemos obtenido es destrucción por las guerras y la destrucción del planeta, explotación laboral etc. es una actitud de que toda realidad depende del cristal con que se mire.

Guidens nos dice que continuamos en modernidad uy que este puede ser un mosnstruo que construye a la vez que destruye con una fuente de creación. Este proceso de cambio coinside con la globalización y todo esta en permanente cambio, guidens nos dice que efectivamente hay cambios pero lo que ocurre no es que hayamos entrado a otra etapa sino que la modernidad se ha radicalizado.

FUENTES DE DINAMISMO DE LA MODERNIDAD

SEPARACIÓN  DEL ESPACIO Y EL TIEMPO
En la modernidad ha cambiado la manera en que nos relacionamos con el tiempo, en las sociedades tradicionales la única manera de relacionarnos con el tiempo estaba conectada al espacio.
Por ejemplo si queríamos decir que es medio día, eso lo identificamos por la posición del sol en el cielo o el caso de decir que es verano lo relacionamos con la temperatura ambiente la luz al amanecer o anochecer .
La manera en que veimaos el tiempo no nos permitía manejar el tiempo como una variable que pudiéramos gestionar o manejar par organizar nuestra vida, de hecho para la gente no era que el tiempo estaba trascurriendo sino que las cosas pasaban cuando las cosas deben ocurrir.
Por ejemplo a que hora parte un taxi colectivo? Cuando se llena el coche, entonces esa actitud de esperar a que ocurra algo es la actitud pre moderna, la actitud moderna seria so se quiere la de una frecuencia como al de los colectivos o trenes o aviones etc. Eso es lo moderno entonces la hora e la que manda
hoy tenemos  mayor control del  tiempo gracias a los relojes, lo que ha ocurrido es la separación del espacio y el tiempo, consiste en que debido a la tecnonlogia hoy podemos medir el tiempo y organizar actividades en torno a los horarios y ya no en torno a hechos naturales o espaciales.
Puedo referirme al tiempo como algo autónomo ya no atado al espacio.
La tecnología también ha hecho posible que el contacto de personas a grandes distancias sea posible sin tener en cuenta el espacio, por ejemplo dos personas hablan por Skype y una esta en una ubicación de día y la otra esta en una ubicación donde es de noche el espacio para ellas es diferente sin embargo esta compartiendo un mismo ahora en la comunicación y eso es separación de espacio y tiempo.


DES ANCLAJE
Si la división entre el espacio y tiempo ha cambiado nuestra relación con el tiempo, el des anclaje cambio nuestra relación con el espacio, el desanclaje dice que ya no estamos fijos a una realidad espacial, es decir antes las interacciones humanas estaban ancladas a una posición geográfica y nuestras relaciones estaban en determinadas por esa ubicación es decir nuestra pareja, nuestro trabajo estudio etc. se referían al lugar donde nos movemos, con la modernidad se desanclan las interacciones sociales, y ahora asestan influenciadas por lo global, 
Mecanismos de desanclare
existen situaciones que generan desanclare es decir que nos ayudan a migrar.
Sistemas expertos son las ideas que las cosas van a funcionar por que están en manos de expertos, estro es posible  gracias a la  confianza ejemplo del avión cuando viajamos no sabemos como va a funcionar todo pero confiamos en que va a funcionar, esta delegación en sistemas expertos facilita el desanclare.
La fe religiosa se traslada a los sistemas expertos
señales simbólicas tiene que ver que estamos usando medios de intercambio que están muy separadas en tiempo y espacio, por ejemplo el dinero, cuando nos envían por western union  ese dinero sirve para pagar la educacion del hijo, este dinero sirve como una señal simbólica, es decir que se conecta personas por medio de este. También existe una cuota de confianza y de esa manera podemos desanclarnos mas.


LA NATURALEZA REFLEXIVA DE LA MODERNIDAD
Guidens tmabe nsion habla de la reflexividad del conocimiento, pero en que consiste?
hemos visto que la modernidad se ha radicalizado y que ha cambiado como nos relacionamos con el 
tiempo y con el espacio con el desanclaje, pero también cambia nuestra forma de conocimiento.

En el mundo tradicional realizabamos una serie de tareas siguiendo la tradición, porque? Porque así nos habían enseñado,  no solo pensemos en las relaciones de trabajo tambien en todo nuestro entorno es decir las relaciones de pareja la crianza de los hijos todo esto lo aprendimos por tradición, 
pero la modernidad es reflexiva es decir que tiende a mirarse a si misma y corregirse como lo haríamos nosotros al mirarnos al espejo. 
Este enfoque frente al conocimiento en que busca evaluarse medirse y hacer los ajuste que nos parezca mejor es lo que guidens llama la naturaleza reflexiva del conocimiento o de la modernidad.
Es como un juego de espejos por ejemplo si alguien publica un nuevo conocimeinto o una novedad academica pronto va a aparecer alguien que lo critique o que lo apoye.
O ademas se generan caminos para la innovacion, eñl techo de lguien va ha convertirse en el piso de otro.

Antes había una reflexividad que se reducía a decidir como ser mejor en torno a la misma tradicional, pero lo que caracteriza a la actitud moderna es que el individuo se siente con el poder de seleccionar lo que tiene que decir, veamos las actitudes religiosas, por ejemplo la moral lo que antes se consideraba mal hoy cambia, 


\end{document}
