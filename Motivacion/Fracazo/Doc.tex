%%%% PROCESAR con PdfLaTeX !!!!!


\documentclass[12pt]{book}
\usepackage{geometry}\geometry{top=2cm,bottom=2cm,left=3cm,right=3cm}
\usepackage{amssymb}
\usepackage{amsmath}
\usepackage{graphicx}
\usepackage{txfonts}




\begin{document}
\thispagestyle{empty}

\begin {center}

\includegraphics[scale=.4]{Logo-fiuba_big.png}

\medskip
UNIVERSIDAD DE BUENOS AIRES

Facultad de Ingenier\'ia

Departamento de Computaci\'on


\vspace{3cm}


\textbf{\large 7510 T\'ecnicas de Diseño}

\vspace{2cm}


Este es un modesto aporte para los alumnos de la f\'acultad de ingenier\'ia  de la UBA de las carreras de licenciatura en an\'alsis de sistemas e ingenier\'ia inform\'atica.
De ninguna man\'era pretende ser una gu\'ia de estudio, ni remplaza las clases presenciales, el material oficial de la catedra esta disponible en el web site de la m\'ateria.
\\
http://materias.fi.uba.ar/7510/

\end {center}


\vspace{2.5cm}

\noindent Autor:\,	Isaac Edgar Camacho Ocampo
 
\noindent Carrera:\,	Licenciatura en An\'alisis de sistemas

\vspace{1cm}

\vspace{1cm}

\noindent Buenos Aires, 2019

\newpage


\tableofcontents

\tableofcontents
\chapter{Introducción}
Cuando hablamos de \'exito o frac\'azo siempre pensamos que son antónimos es decir que tienen significado opuesto, pero en realidad no es asi.Veamos los siguientes ejempos
\section{Conocimientos previos}

Cómo Planificar Estratégicamente tu 2015
Published on 2014 M12 23

Nelson Portugal
Nelson PortugalFollow
Consultor en Desarrollo Personal
Like19
Comment10
0
Si vas a ir a ver a tu amigo, tienes que saber por qué estás yendo ¿cuál es tu objetivo? ¿Cuál es tu meta?” – decía Miguel Ángel Cornejo en una de sus conferencias.

Tenía 15 años y era la primera vez que escuchaba la palabra “meta”.

Yo, sin entender a que se refería exactamente, escribía todo lo que iba escuchando sobre las supuestas metas.

Planificación Estratégica
Con el tiempo aprendí conceptos más sofisticados como las metas S.M.A.R.T, el Efecto Dominó, el enfoque de “The One Thing”, el sistema de planeación de Anthony Robbins, la ley de Pareto y su aplicación en el uso del tiempo, etc.

Pero, sobretodo, aprendi el efecto que tiene la planificación estratégica en la forma en que vivimos nuestra vida y en nuestra capacidad de convertir nuestros sueños en realidad.

Aprendí, también, porqué la mayoría de personas no logra sus objetivos y, pese a que tiene el deseo de cumplirlos, termina postergándolos año tras año.

Si estás leyendo estas líneas es porque hay algo en tu vida con lo que no te sientes satisfecho(a) y te gustaría mejorar.

Tal vez son tus finanzas, tu peso, tus relaciones, tu carrera profesional, tu relación con tu familia, amigos o con tu creador.

No importa qué es aquello que te gustaría lograr o mejorar, en las próximas líneas aprenderás cómo puedes crear un plan estratégico para conseguirlo.


Para ello dividiré este artículo en tres partes.

En la primera parte: revelaré los tres errores más comunes que cometen la mayoría de personas al trazar metas y cómo tú puedes evitarlos.
En la segunda parte: trabajaremos juntos en establecer objetivos específicos para tu vida de forma tal que, finamente, puedas hacerlos realidad.
Y en la tercera parte: conocerás un concepto clave en la planificación de objetivos: la fuerza invisible que te permitirá superar los obstáculos que se presentan en el camino a convertir un deseo en un hecho.
Si ya has decidido leer por completo este artículo y aplicar, paso a paso, lo que revisaremos en él, felicitaciones!

… acabas de pasar del asiento trasero al asiento del conductor del vehículo de tu propia vida.

Si aún no lo has hecho y estás pensando simplemente en “escanear” este artículo y regresar a tu día a día sin crear un plan específico para los próximos años de tu vida, entonces permíteme recordarte que:

No es lo que sabes sino lo que haces, lo que cambia tu vida. RTuitéalo

Digo esto porque no importa lo que leas, veas o escuches sino lo que haces con aquello que aprendes.

Y porque diciembre del próximo año llegará lo quieras o no, la diferencia entre estar viviendo la vida que tú quieres y una vida con arrepentimientos, frustraciones y postergaciones será lo que hagas en este momento.

Toma acción en lo que aprenderás en este artículo y tendrás un plan real, específico y efectivo para acercarte a una vida en tus propios términos.

Úsa este artículo como simple entretenimiento intelectual (leerlo solo por saber) y verás desde la banca de suplentes como termina el juego de tu vida.

Si ya tengo tu atención y compromiso, entonces empecemos.

1. Los 3 errores más comunes en la planificación y cómo puedes evitarlos
Antes de revisar los tres errores más comunes y aprender qué hacer para evitarlos, debemos entender qué significa, específicamente, “trazarnos metas”.

Decir “me gustaría tener mi propio departamento” no es trazar una meta, es simplemente expresar un deseo.

Trazarnos metas significa articular en una oración, específicamente, aquello que queremos lograr, crear o experimentar en el futuro.

Significa escribir esta oración en la hoja de tu cuaderno, en la pizarra de tu oficina, en la nota de tu iPhone y revisarla todas las semanas para verificar tu progreso y recordarte a ti mismo lo que realmente quieres en la vida.

Lamentablemente al formular nuestras metas cometemos tres errores que frustran la posibilidad de que las logremos.

Error 1:
La mayoría de personas al trazarse metas suele escribir frases como “quiero más dinero”, “quiero bajar de peso”, “quiero ascender”, “tener mi propio negocio”, “pasar mas tiempo con mi familia”, “tener más tiempo para mi mismo(a)”.

Lamentablemente, estas personas nunca lograrán lo que quieren en la vida porque ninguna de estas metas es suficientemente específica para poder definir estrategias efectivas para conseguirlas.

Por el contrario, incluso podrían ser comparadas fácilmente con reclamos que buscan expresar disconformidad en lugar de intención de cambio.

¿Tiene sentido?

Decir “quiero mas dinero” es similar a decir “no me alcanza lo que gano ahora”. Decir “quiero bajar de peso”, a “no me gusta como me veo”.

Y aunque “quiero tener mi propio negocio” podría expresar cierto nivel de deseo, el hecho de no saber qué tipo de negocio, la cantidad de tiempo y el capital necesario para hacerlo realidad dificulta la toma de acción.

Entonces, ¿como debemos definir nuestros objetivos?

De forma específica.

Por ejemplo:

Para antes de fin del 2015 estaré ganando \$2500 mensuales netos
Para antes de Julio del 2015 tendré un % de grasa menor al 15%
¿Notas la diferencia?

¿Cuándo se te hace mas sencillo encontrar estrategias para conseguir tu meta? ¿Cuando dices “ganaré más dinero” o cuando dices “ganaré \$2500 mensuales para antes de fin de año”?

¿Cuando te sientes más motivado(a) a tomar acción? ¿Cuando dices “tendré mi propio negocio” o cuando dices “para antes de fin de año tendré 20 clientes en mi negocio de consultorías de marketing para empresas”?

Y ¿cuando será más sencillo medir tu progreso y saber si vas por buen camino? ¿Cuando dices “bajaré de peso” o cuando dices “para Julio del próximo año tendré solo 20% de grasa”?

En todos los casos podrás ver que será más sencillo encontrar estrategias, medir el progreso y lograr tus metas cuando éstas son realmente específicas.

Lamentablemente, muy pocas personas articulan sus objetivos de esta forma.

De hecho, en un artículo anterior llamado “deja de ponerte metas, empieza a tomar acción” escribí una de las barreras psicologías que debemos superar para definir metas específicas: el miedo al fracaso.

Verás, como detallo en aquel artículo, definir metas específicas nos permitirá saber exactamente lo que queremos conseguir – lo que hará más sencillo ir tras ese objetivo -, pero también sabremos con facilidad cuando no lo hemos logrado… cuando hemos fracasado.

Y es el miedo natural que le tenemos al fracaso lo que hace que no nos atrevamos a definir específicamente lo que queremos conseguir.

Sin embargo, tú y yo hemos aprendido que cuando menos sencillo es hacer algo, con más razón debemos hacerlo.

Por eso es que debemos atrevernos a definir nuestras metas de manera específica. Porque, además, es la única forma de hacerlas realidad.

La pregunta es: ¿cómo, debemos plantearlas?

A continuación te escribo tres ejemplos de metas realmente especificas:

Ahorraré el 10% de todos mis ingresos durante el próximo año para completar la inicial de mi primer departamento
Empezaré mi negocio de consultorías independientes y conseguiré al menos 10 clientes pagados hasta antes de junio del próximo año.
Me matricularé a las clases de salsa y asistiré al menos dos veces a la semana durante el verano.
Mira cómoo estas metas son completamente diferentes al típico “quiero un departamento, poner mi negocio propio y aprender a bailar salsa.”

Al ser específicas podrás saber qué es exactamente lo que tienes que hacer para conseguirlo y tendrás claro cuando te estás acercando o alejando de ellas.

En pocas palabras, podrás, finalmente, alcanzarlas.

Entonces, ¿estás listo(a) para tener metas específicas en tu vida y hacer lo necesario para conseguirlas?

Si es así, pasemos a conocer y solucionar el segundo error más común.

Error 2:
Ahora que ya sabes la importancia de definir metas específicas quiero que conozcas el segundo error que suelen cometer las personas al trazarse metas: >enfocarse en solo una área de sus vidas.

Es común escuchar a las personas decir que les gustaría ascender, ganar mas dinero, poner su propio negocio, comprarse un auto, una nueva camioneta, una casa o un departamento.

Pero aún es poco común escuchar a alguien decir que quiere desarrollar nuevas amistades, aprender a tocar un instrumento de musica, fortalecer su relación familiar, correr una maratón o llevar alimento a quienes no lo tienen.

¿Por qué?

Porque nos hemos acostumbrado a tener metas para nuestra vida profesional, específicamente nuestras finanzas, y nos hemos olvidado de buscar activamente mejorar otras áreas de nuestra vida, igual de importantes como…

Nuestra familia, nuestras amistades, nuestro nivel de contribución, nuestra relación de pareja, nuestra salud, nuestro desarrollo personal, etc.

Queremos ganar más dinero, pero olvidamos aprender a escuchar, nutrir relaciones, divertirnos, cuidar nuestro cuerpo, aumentar la pasión en nuestra relación y muchos otros aspectos importantes en nuestra vida.

Por esta razón es que luego de cinco, diez o veinte años de esfuerzo finalmente conseguimos ese ascenso, departamento, casa, camioneta o negocio propio solo para darnos cuenta que, en el camino, perdimos lo que mas queríamos en nuestra vida: a nuestra(o) novia(o), nuestros hermanos(as), nuestra salud…

Y de esta forma nos damos que cuenta que de nada habrá servido lograr nuestras más grandes metas financieras si no consideramos también nuestras metas de salud, relación, contribución y demás.

Por eso es que en lugar de definir una, dos o tres metas sobre un mismo aspecto – o categoría – es importante que definamos tres objetivos para cada una de las categorías importantes en nuestra vida – usualmente ocho:

1) Finanzas, 2) salud, 3) relación de pareja, 4) familia, 5) espiritual, 6) amistades, 7) carrera profesional y 8) diversión

…lo que haría un total de 24 metas durante el año.

Si, podría parecer imposible o incluso poco eficiente tener esa cantidad de metas. Sin embargo, esta es la única, y mejor, forma de progresar en todos los aspectos de nuestra vida.

Para que puedas tener un mejor entendimiento sobre este concepto y puedas identificar las categorías más relevantes para tu vida, mira los siguientes dos vídeos que he tomado prestado del entrenamiento gratuito de Vida Extraordinaria:



En el video anterior viste la importancia de reconocer que nuestra vida esta compuesta por dos áreas: personal y profesional.

Ahora verás como cada una de éstas áreas alberga diferentes categorías dentro de ellas.



Por esta razón es que en nuestra planificación de metas debemos considerar todos los aspectos de nuestra vida.

De hecho, en la segunda parte de este artículo pondremos en practica este concepto.

Por ahora, vayamos a tercer problema mas común en la planificación de nuestra vida y cómo solucionarlo.

Error 3:
¿Has escuchado alguna vez la siguiente frase: “la diferencia entre un sueño y una meta es que la meta tiene una fecha y el sueño no”?

Tiene sentido ¿verdad?

Lamentablemente esta frase solo cubre la superficie del asunto, porque no importa que la meta tenga fecha, si no definimos actividades que realizaremos en un día específico y a una hora determinada, jamás la lograremos.

Y es justamente el hecho de que la mayoría de personas al planificar sus metas, no consideran la día y la hora en la cual realizarán las actividades que harán posible el cumplimiento de las metas el tercer error más común – y grande – que cometemos al planificar nuestro futuro.

Míralo de esta forma:

Si un sueño tiene fecha, es una meta, pero si tiene un horario, es inevitable. - RTuitéalo

Imagina, por ejemplo, que tu sueño sea correr una maratón. Tu objetivo sería correr una maratón para Julio del próximo año.

Para hacer de esta meta algo inevitable – es decir una realidad – necesitamos identificar las actividades principales que debemos realizar y establecer un horario para ellas.

Lo cual luciría algo así:

Durante los próximos seis meses correré tres veces por semana – veinte minutos el lunes a las 7am, cuarenta minutos el miércoles a las 8pm y una hora el sábado a las 9am.

Ahora piensa en esto ¿Qué probabilidad habrá de que puedas corras una maratón si corres tres veces por semana por los próximos seis meses?

Preguntado de otra forma ¿Qué probabilidad habrá de que logres tus objetivos si estableces horarios para las actividades principales que debes realizar para lograrlos?

Virtualmente tendrías una probabilidad del 100%.

Esto es a lo que yo llamo “éxito inevitable”.

Sin embargo, la mayoría de personas nunca llega a esta etapa de la planeación y se quedan únicamente en lo que quieren conseguir, dejando de lado aquello que deben realizar para lograrlo.

Pero no tú, pues ya conoces cómo hacer de tus sueños metas y de tus metas, logros inevitables: estableciendo horarios.

¿Tiene sentido?

Si lo tiene entonces ya has descubierto los tres errores más comunes que cometemos al definir nuestras metas y, lo mas importante, sabes cómo evitarlos.

2. Hagamos ahora un plan para tu próximo año – y el resto de tu vida
Ahora solo debes:

Identificar las categorías más importantes de tu vida.
Definir tres objetivos específicos para cada uno de ellos.
Establecer un horario para las actividades que te permitirán conseguirlos.
Y eso es lo que haremos a continuación. Para ello toma un lápiz y papel o abre un nuevo documento en excel ahora.

¿Ya lo tienes? Empecemos…

1. Identica las categorías más importantes de tu vida

¿Qué categorías – o aspectos – son importantes en tu vida?

Estas podrían ser:

Salud
Familia
Relación de pareja
Carrera profesional
Negocio propio
Finanzas
Diversión
Amistades
Desarrollo personal
Espiritual
Contribución
Hogar
Al principio podría parecer que todas las categorías son importantes y aunque ése podría ser el caso, quiero que te tomes al menos tres minutos para elegir aquellas que forman parte de tu día a día y te gustaría desarrollar constantemente.

Algunos incluirán el “hogar” como categoría porque son ellos ven, personalmente, la limpieza, decoración y les gustaría renovar la casa o departamento con frecuencia.

Mientras que otros no lo harán porque viven en casa o departamento alquilado y el “hogar” no es un categoría en la cual desean enfocarse.

De la misma forma habrán personas que no tienen un negocio propio ni esperan tenerlo por los próximos tres, cinco o diez años por lo que esta categoría no es relevante para sus vidas en estos momentos, pero sí considerarán “carrera profesional” como un aspecto importante.

Lo importante es que selecciones las categorías más relevantes para ti en esta etapa de tu vida, pues definirás objetivos específicos dentro de cada una de ella para que puedas pasar de donde estás ahora a donde quieras estar.

Ejemplo:

De pesar 120kg, a 85kg.
De ganar $3 mil al mes a $9 mil
De no hablar con tu hermano a tener una buena relación con él
De no saber ingles a tener certificación
De no tener tiempo para ti a ir al teatro todas las semanas, etc.
Y por si te estás haciendo la siguiente pregunta: ¿por qué es importante definir categorías en lugar de tener diez, once o doce metas independientes?

…te contaré una pequeña historia que te permitirá obtener la respuesta por ti mismo(a).

Lea con atención, pues es sumamente importante.

Imagina que llegas de trabajar un día y encuentras en tu casa que todas tus pertenecías, absolutamente todas, están una sobre otra en una esquina de tu sala – tus platos, tu televisor, tu ropa, tus cepillo de dientes, tu jabón, tus cuadernos, tus lapiceros, tu billetera, tus vasos, etc.- y tienes que ordenarla antes de irte a dormir.

¿Como te sentirías? ¿Sabrías por donde empezar? ¿Cuánto tiempo te demorarías? Y si de pronto, cuando vas por la mitad, vuelve todo a la esquina de tu sala ¿podrías volver a empezar?

Recuerda tus respuestas por un momento y ahora imagina esta otra situación.

Llegas a tu casa de trabajar y todas tus pertenecías están, al igual que la situación anterior, en una esquina de tu sala debes ordenarlas.

Pero esta vez tienes tienes un armario grande cuatro separaciones con las siguientes descripciones: útiles de aseo, accesorios personales, material de cocina y ropa.

Tu trabajo es solo colocar cada pertenencia en la sección que le corresponde.

¿Como te sentirías entonces? ¿Se te haría más sencillo empezar? ¿Terminarías más rápido ahora que tienes secciones específicas para cada cosa?

La ciencia dice que sí.

Al tener ya una forma de agrupar las decenas de pertenecías en sólo cuatro secciones – o categorías – hace que tu cerebro se organice más rápidamente y sea más efectivo al administrarlas.

Incluso cuando no sea decenas sino centenas o miles de pertenecías tu mente sabrá a que sección pertenece cada una de ellas por lo que se le hará más sencillo organizarlas porque limitará el número de opciones.

Por el contrario, si no tienes estas divisiones, tu mente estaría sobrecargada, te sentirías presionado(a), angustiado(a) y difícilmente podrías organizarlo correctamente.

De la misma forma cuando intentas organizar tu vida sin tener categorías específicas a las cuales le pertenece cada situación, cada meta, cada decisión o actividad, te sentirás agobiado(a) al tratar de lograr mejorar en cada una de ellas.

Piensa, por ejemplo, ¿cómo te de sentirías si supieras que no importa lo que ocurra, solo existen ocho aspectos de tu vida en los que debes colocar tu atención, enfoque, energía, tiempo y dedicación?

¿Cuánto más sencillo sería para ti saber, precisamente, qué es lo que quieres lograr en cada una de estas ocho categorías? ¿Cuánta más claridad tendrías?

¿Con cuánta más facilidad podrías organizarte?

Y, sobretodo, cuánto más eficiente sería tu toma de decisiones si supieras en qué no enfocarte? Podrías fácilmente eliminar las distracciones que no pertenecen a ninguna categoría importante para ti.

De eso se trata poder seleccionar las categorías que formarán tu vida. Ya sean seis, ocho o doce, te ayudarán a organizarte de mejorar manera y ver tu vida desde una nueva, y mejor, perspectiva.

Si no las has seleccionado ahora, vuelve a ver la lista que esta algunos párrafos arriba y hazlo ahora.

¿Ya tienes tus seis, siete, ocho o nueve categorías?

Muy bien!

2. Define tres objetivos para cada categoría

Ahora es momento de que escribas tres objetivos específicos para cada una de ellas para que sepas precisamente lo que quieres lograr, experimentar, crear o cambiar.

¿Qué te gustaría lograr en tu categoría de salud? ¿Te gustaría hacerte un chequeo cada 6 meses? ¿Reducir tu % de grasa a 18%? ¿Correr 10km en la próxima carrera de tu ciudad? ¿Ser más flexible?

Elige tres objetivos que son importantes para ti en este aspecto específico de tu vida y piensa en lo emocionante que sería ir tras ellos durante el próximo año.

Ahora define tres objetivos para tu categoría de relación familiar.

¿Te gustaría salir a almorzar en familia al menos una vez por semana? ¿Hacer un viaje internacional o nacional durante el próximo año? ¿Encontrar un hobby o actividad que puedas realizar con toda tu familia? ¿Tal vez ir a clases de algun curso o deporte juntos?

Recuerda que el objetivo debe ser específico y debe existir una forma de saber que ya los has logrado.

Decir “pasar más tiempo con mi familia”, como habíamos visto anteriormente, no es suficiente.

Debes escribir, específicamente, cuánto tiempo, que días y que harás durante ese momento.

Tómate unos treinta minutos y haz esto para todas las categorías que habías seleccionado para tu vida: tu carrera profesional, negocio propio – si lo tuvieras o quisieras tenerlo – relación de pareja, diversión, etc.

Deja de leer, toma un lapiz y papel, y hazlo ahora.

…

¿Ya lo hiciste?

Recuerda que leer acerca de esto, conocerlo o incluso entenderlo no cambiará tu vida.

Realmente quiero que tomes un lápiz y papel, abras un excel o una nota en tu Laptop/Iphone y escribas los 18, 21 o 24 objetivos para el próximo año para tus seis, siete, ocho o nueve categorías.

…

En caso te gustaría que te ayude personalmente con este proceso y estás comprometido(a) en aprender cómo optimizar el uso de tu tiempo, energía y emociones, entonces sé parte del próximo Workshop de Planeamiento Estratégico en vivo.

Para leer los detalles, requisitos y costos haz click aquí ahora.

Este programa es un Workshop de dos días (Miércoles y Jueves de 6pm a 10pm) y aunque definitivamente no es accesible para todos, para los(as) profesionales, ejecutivos(as) y empresarios(as) que tengan la capacidad económica está es una gran opción.

…

Ahora que ya tienes definidos los objetivos de cada aspecto de tu vida, tomate unos cuantos segundos para imaginarte cómo seria tu vida si ya fueran una realidad.

¿Como te sentirías si aquello que hoy ves en papel – o pantalla – dentro de doce meses ya fuera parte de tu vida?

Dibuja esa imagen en tu mente y siente la emoción recorrer por cada parte de tu cuerpo.

Definitivamente no será fácil, pero te digo algo: valdrá la pena.

Si has llegado hasta aquí es porque eres de las personas que busca soluciones. Busca mejores formas de hacer las cosas para conseguir mejores resultados en su vida.

Por esa razón quiero continúes conmigo en este viaje e identifiquemos ahora lo que debemos hacer para convertir esos objetivos en una realidad.

Hace algunos minutos revisamos el tercer error más común al planificar los objetivos ¿recuerdas cuál era? No establecer horarios para las actividades principales ¿verdad?

Si dejaras de leer este artículo en esta sección lo mas probable seria que tus metas se quedarán únicamente en un simple “sería bueno lograrlo” porque no tendrías un plan real para convertirlo en “lo lograré”.

3. Establece un horario para las actividades que te permitirán conseguirlos

Por eso es que quiero que te tomes, al menos, diez minutos por cada categoría para que definas tres actividades principales que deberás establecer en tu horario semanal para lograr cada objetivo.

Por ejemplo,

Si lo que quieres es conseguir un trabajo que te guste y te pague bien lo que debes hacer – luego de definir el área y la industria especifica en la cual quieres trabajar, la cantidad de dinero que esperas ganar al mes y los beneficios adicionales que te gustaría obtener – podría ser lo siguiente:

Primera actividad: Crear un documento en excel con todos los posibles puestos de trabajo, número de contacto y referencias, y alimentara base de datos todas las semanas.
Segunda actividad: Contactar estratégicamente a cada persona para coordinar una posible llamada, reunión, almuerzo o cena.
Tercera actividad: Investigar acerca de la industria, área, perfil de puesto y/o empresa en la que deseas trabajar para conocer las mejores prácticas, idear formas de trabajo más eficientes y poder aumentar tu valor de mercado.
Nota: Tal vez las actividades que tú hubieras elegido sean menos efectivas o más efectivas que las que acabo de escribir yo. Sin embargo, lo importante es realizarlas, ver los resultados y modificarlas si fuera necesario. También es importante que se las muestres a una persona que ya haya conseguido el resultado que tu deseas y preguntarle qué opina de esas actividades y si conoce alguna actividad más efectiva que podría recomendarte.

Luego de definir las tres principales actividades debes establecerlas en tu horario para lo cual necesitas definir la siguiente información:

Cual será el procedimiento a seguir – esto lo puedes guardar en una nota o documento para hacer más fácil la repetición o incluso, eventualmente, poder delegarla.
Definir cuanto tiempo le dedicaras a este proceso/actividad.
Seleccionar la frecuencia, el día y hora de la semana en el realizaras la actividad.
Al hacer este ejercicio podrías pensar que “el tiempo no te alcanza” para lo cual quiero que recuerdes lo siguiente:

Puedes delegar una actividad a un tercero – ya sea empresa o persona.
Puedes aprender cómo optimizar el uso de tu tiempo en el Workshop en vivo que realizo.
No necesitas trabajar en todos tus objetivos cada semana, puedes trabajar en alguno de ellos una vez al mes.
Cuando estás realmente comprometido(a), siempre existe una forma de lograrlo.
Lo importante es que tengas en claro qué actividad debes realizar y en qué momento la harás para poder conseguirlo.

Como te había dicho antes, esto es a lo que yo llamo “éxito inevitable” y tú puedes conseguirlo. Pero necesitas tener definido lo siguiente:

Las categorías importantes de tu vida.
Los tres objetivos del próximo año para cada categoría.
Las tres actividades principales para casa objetivo.
El proceso, la frecuencia y la fecha en la cual realizaras cada actividad.
Haz esto y serás parte del 0.01% que sabe exactamente lo que quiere y tiene un plan efectivo para conseguirlo.

Falla en tomar acción y estarás cometiendo el error de omisión mas grande de los próximos cinco, diez o veinte años de tu vida.

¿Sabes por qué?

Porque si tu no tienes un plan para tu vida, otro persona lo tendrá.

¿No nos pasa eso acaso cuando no sabemos lo que haremos un día específico y de pronto un(a) amigo(a) cercano nos pregunta: “¿me acompañas a comprar unas cosas?”?

Está bien hacerlo un sábado por la tarde o un domingo por la mañana, pero ¿donde terminarás si vives así durante un año entero?

Lo mas probable es que no sea en un lugar donde te gustaría estar.

Y si estás pensando en que tú no puedes tener un plan, porque Dios tiene preparado algo para ti, quiero compartir contigo una pequeña historia.

Cuando tenía 13 años había escuchado ya varias veces a las personas decir “si Dios quiere” cuando le preguntaban si irían a una cena, un evento o una reunión en general.

Por esta razón recuerdo que cuando mi papá me preguntó si estudiaría ingles yo respondí “si Dios quiere” a lo que él me respondió: “Dios sí quiere, Dios quiere lo mejor para ti, tú asegúrate de hacer lo que tienes que hacer para conseguirlo”.

Tal vez no lo dijo con estas mismas palabras, pero recuerdo que la lección que me quedó de esta pequeña conversación fue que Dios está de nuestro lado, pero somos nosotros quienes debemos poner el esfuerzo y la dedicación para tomar lo que nos pertenece: una vida de logros y satisfacción.

3. Utiliza la fuerza invisble para tomar acción
Quiero terminar este texto contándote una de los conceptos más importantes que he aprendido en los últimos años de mi vida.

Como te dije al iniciar este artículo, existe una fuerza invsible que te permita superar cualquier barrera que se te presente en el camino.

Esta fuerza está representada por la razón por la cual quieres conseguir tus objetivos. En otras palabras: “el por qué” de lo que haces y quieres conseguir.

No es lo mismo, por ejemplo, querer ir a tu casa porque te gustaría ver televisión, que querer ir a tu casa porque un familiar ha regresado de un viaje de más de cinco años.

El objetivo es el mismo: ir a tu casa. La razón es completamente diferente. La pregunta es ¿qué razón hará que digas que no cuando te ofrezcan pasar la tarde en la casa de otra persona?

De la misma forma, si te gustaría aumentar tus ingresos porque “sería bueno” podrías distraerte en el camino y no tomar la acción necesaria a diferencia de si tu “por que” es porque quieres…

Remodelar la casa de tu madre, cambiar de colegio a tus hijos, pagarle una mejor universidad, pagar tus deudas y vivir con mayor tranquilidad, poder regarle lo que siempre a querido tu novio(a) o esposo(a) y nunca han podido pagar, etc.

Por ello esta es la última lección que quiero dejarte el día de hoy:

Mientras mayor sea tu “por qué”, más resistencia podrás soportar en el camino a lograr tus objetivos. - RTuitéalo

Enfócate entonces, no solo en los objetivos que quieres lograr sino también en por qué quieres lograrlo. Tómate algunos minutos adicionales ahora y escribe al costado de cada objetivo – y en dos líneas – por qué quieres hacer realidad ese objetivo.

¿Cómo te sentirías? ¿A quién podrás inspirar? ¿A quién podrás ayudar? ¿Cómo tu vida será mejor? ¿Qué podrás comprar, experimentar, lograr o disfrutar si este objetivo ya fuera una realidad?

Escribe tus respuestas al costado de cada objetivo y completa así tu plan estratégico para el próximo año.

Lo que debes hacer ahora es imprimir este documento, en caso lo hayas hecho en tu iPhone o laptop, y pegarlo en un lugar de tu cuarto donde puedas verlo todos los días del año.

También puedes escribir en tu agenda o programar tu calendario en tu celular las principales actividades que has identificado para el logro de tus objetivos.

Puedes seleccionar un color diferente para cada categoría y reconocer así que estás progresando no solo en una sino en todos los aspectos de tu vida.

Algo similiar a la imagen que ves a continuación:


En el Workshop de Planeamiento Estratégico vemos a profundidad este concepto y desarrollamos un sistema de planeación semanal que nos permite revisar, medir y mejorar nuestro plan anual.

De esta forma podemos asegurarnos de que lo que sea que hayamos escrito en papel sea realizado en la realidad de la manera más efectiva posible.

De nuevo, si te gustaría ser parte del programa en vivo haz click aquí para revisar los detalles, costos y requisitos.

De lo contrario, vuelve a revisar este artículo por completo y asegúrate de aplicar, paso a paso, lo que hemos revisado en él. Recuerda que diciembre del próximo año llegará lo queramos o no.

Donde estemos en ese momento dependerá única y exclusivamente en lo que hagamos en este preciso instante.

Tómate el tiempo para planficar tu año y estarás en camino a vivir la vida que siempre soñaste.

Un abrazo,

Nelson

PD: Este artículo fue publicado originalmente en: http://www.nelsonportugal.com/planificacion-estrategica

Para recibir los próximos artículos de la Serie 2015 ingresa tu email en este enlace web: http://www.nelsonportugal.com/serie-2015
\section{Estado del arte}
\chapter{Fundamentos teóricos}
\section{Teoría clásica}
\subsection{Definición de variables}
\subsection{Pruebas y refutaciones}
\section{Hipótesis}
\chapter{Resultados}
\section{Simulación de resultados}
\subsection{Suposiciones}
\subsection{Modelos}
\section{Resultados preliminares}
\section{Resultados postprocesados}
\subsection{Valores atípicos}
\subsection{Correlaciones}
\chapter{Conclusiones}
\end{document}
