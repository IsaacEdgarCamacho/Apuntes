\documentclass{article}
% pre\'ambulo

\usepackage{lmodern}
\usepackage[T1]{fontenc}
\usepackage[spanish,activeacute]{babel}
\usepackage{mathtools}

\title{Epistemologia}
\author{Isaac Edgar Camacho}

\begin{document}
% cuerpo del documento

\maketitle

Las desventuras del conocimiento.
\\
Las desventuras del conocimiento científico  Klimovsky CAP 1	
El concepto de ciencia
Ciencia, conocimiento y método científico

la ciencia es fundamentalmente un acopio de conocimiento, que utilizamos para comprender el mundo y modificarlo. 
¿qué condiciones debe cumplir? Según lo expone Platón tres son los requisitos que se le deben exigir para que se pueda hablar de conocimiento: creencia, verdad y prueba. En primer lugar, quien formula la afirmación debe creer en ella. Segundo, el conocimiento expresado debe ser verdadero. Tercero, deberá haber pruebas de este conocimiento.

Si no hay creencia, aunque por casualidad haya verdad y exista la prueba, pero ésta no se halle en poder de quien formula la afirmación, no podremos hablar de conocimiento. Tampoco podremos hacerlo si no hay verdad, porque no asociamos el conocimiento a sostener lo que no corresponde a la realidad o a los estados de cosas en estudio. Y aunque hubiese creencia y verdad, mientras no exista la prueba se estará en estado de opinión mas no de conocimiento.

En la actualidad ninguno de los tres requisitos se considera apropiado para definir el conocimiento científico.  La concepción moderna de éste es más modesta y menos tajante que la platónica, y el término prueba se utiliza para designar elementos de juicio destinados a garantizar que una hipótesis o una teoría científicas son adecuadas o satisfactorias  Ya no exigimos del conocimiento una dependencia estricta entre prueba y verdad. Sería posible que hubiésemos probado suficientemente una teoría científica sin haber establecido su verdad de manera concluyente, y por tanto no debe extrañar que una teoría aceptada en cierto momento histórico sea desechada más adelante.

las hipótesis y teorías científicas se formulan en principio de modo tentativo por lo cual la indagación en búsqueda de pruebas no supone una creencia intrínseca en aquéllas Además puesto que no todo conocimiento es conocimiento científico.

Según algunos epistemólogos lo que resulta característico del conocimiento que brinda la ciencia es el llamado método científico un procedimiento que permite obtenerlo y también, a la vez justificarlo.


Aquí hablar de método científico sería referirse a métodos para inferir estadísticamente construir hipótesis y ponerlas a prueba Si es así, el conocimiento científico podría caracterizarse como aquel que se obtiene siguiendo los procedimientos que describen estas estrategias básicas.


Disciplinas y teorías científicas

Una teoría científica, en principio, es un conjunto de conjeturas, simples o complejas, acerca del modo en que se comporta algún sector de la realidad. Las teorías no se construyen por capricho, sino para explicar aquello que nos intriga, para resolver algún problema o para responder preguntas acerca de la naturaleza o la sociedad.

Disciplinas: Conjunto de reglas o normas cuyo cumplimiento de manera constante conducen a cierto resultado

lenguaje y verdad: hemos dicho que el conociemiento se expresa por medio de afirmaciones en el sentido linguistico, ademas el conociemiento es privado de quien lo crea y solo pasa a ser propiedad social si este lo comunica atravez de un leguaje.
Tambien dijimos que la verdad se puede aplicar a enunciados y no a terminos primitivos 
ejemplo tiene sentido hablar de la falsedad o verdad de “el cielo esta azul” y no aplicarlo a los terminos cielo o a azul

Platón exigía, como ya señalamos, que para que un enunciado exprese conocimiento debe ser verdadero. Pero la cuestión es mucho más difícil de lo que aparenta. Como veremos más adelante, una teoría científica puede expresar conocimiento y su verdad no estar suficientemente probada. Dado que el problema radica en la esquiva significación de la palabra "verdad", tendremos que aclarar en qué sentido la utiliza.


En el lenguaje ordinario la palabra "verdad" se emplea con sentidos diversos. 
    1. Por un lado parece indicar un tipo de correspondencia o isomorfismo entre nuestras creencias y lo que ocurre en la realidad. Dicho con mayor precisión: entre la estructura que atribuimos a la realidad en nuestro pensamiento y la que realmente existe en el universo.
    2. Pero a veces parece estar estrechamente ligada a la idea de conocimiento, lo cual podría transformar la definición platónica en una tautología: decimos, en medio de una discusión esto es verdad o "esto es verdadero" para significar que algo está probado. 
    3. En otras ocasiones, curiosamente, verdad se utiliza no en relación a la prueba sino a la creencia. Decimos: Esta es tu verdad, pero no la mía",con lo cual estamos cotejando nuestras opiniones con las del interlocutor
    4. hay un cuarto y muy importante sentido de la .palabra "verdad": decir, por ejemplo, que una proposición matemática es verdadera significa decir que es deductible a partir de ciertos enunciados de partida, fijados arbitrariamente por razones que luego examinaremos.
       
La primera acepción es en principio la que resulta de mayor utilidad. Proviene de Aristóteles quien la presenta en su libro Metafísica, y por ello se la llama "concepto aristotélico de verdad Se funda en el vínculo que existe entre nuestro pensamiento, expresado a través del lenguaje y lo que ocurre fuera del lenguaje, en la realidad. Aristóteles se refiere a esta relación como adecuación o "correspondencia entre pensamiento y realidad. De allí que a la noción aristotélica se la denomine también concepción semántica de la verdad, pues la semántica, como es sabido, se ocupa de las relaciones del lenguaje con la realidad, que está más allá del lenguaje. La acepción aristotélica nos resultará muy conveniente para comprender qué es lo que hay detrás de ciertas formulaciones del método científico y en particular del llamado método hipotético deductivo. Sin embargo, no todos los filósofos epistemólogos o científicos estarían de acuerdo en utilizar la palabra Verdad" con la significación aristotélica. En el ámbito de las ciencias formales, como la matemática

En lo que sigue centraremos nuestra discusión en el papel de la ciencia entendida como conocimiento de hechos, y en tal sentido la matemática, aunque también será analizada, al igual que la lógica, será considerada como una herramienta colateral que sirve a los propósito de las ciencias fácticas, cuyo objetivo es, precisamente, el conocimiento de los hechos. Sin embargo, ésta es una palabra que se emplea con muchos significados, y será necesario aclarar cuál de ellos adoptaremos nosotros. 
Diremos que un hecho es la manera en que las cosas o entidades se configuran en la realidad, en instantes y lugares detenninados.

Será un hecho, por tanto, el que un objeto tenga un color o una forma dadas, que dos o tres objetos posean determinado vínculo entre sí o que exista una regularidad en acontecimientos de cierta naturaleza. En los dos primeros casos hablaremos de hechos singulares, pero al tercero lo consideraremos un hecho general. Cuando una afirmación que se refiere a la realidad resulta verdadera, es porque describe un posible estado de cosas que es en efecto un hecho. No utilizaremos la palabra hecho", por tanto, para la matemática, la lógica y las ciencias formales en general.

La noción aristotélica de verdad no tiene ingrediente alguno vinculado con el conocimiento. Una afirmación puede ser verdadera sin que nosotros lo sepamos, es decir, sin que tengamos evidencia de que hay correspondencia entre lo que describe la afirmación y lo que realmente ocurre. También podría ser falsa, y nosotros no saberlo.


en el lenguaje ordinario hay cierta inclinación a suponer que si hay verdad hay también conocimiento y prueba, lo cual podría generar graves malentendidos. Por ejemplo, no nos permitiría comprender correctamente la fundamental noción de afirmación hipotética o hipótesis. Como veremos luego, quien formula una hipótesis no sabe si lo que ella describe se corresponde o no con los hechos. La hipótesis es una conjetura, una afirmación cuyo carácter hipotético radica en que se la propone sin conocimiento previo de su verdad o falsedad. Uno de los problemas que plantea la investigación científica es el de decidir con qué procedimientos, si es que los hay, podemos establecer la verdad o la falsedad de una hipótesis.


Verificación y refutación

Se debe recurrir a palabras más adecuadas para señalar que se ha probado la verdad o la falsedad de un enunciado -> verificado o refutado. Un enunciado verificado es aquel cuya verdad ha sido probada. Si queremos decir que se ha establecido su falsedad diremos que el enunciado está refutado. Si una afirmación está verificada, entonces necesariamente es verdadera, aunque otra afirmación puede ser verdadera sin estar verificada, lo mismo ocurre con la refutación.

Filosofía de la ciencia, epistemología, metodología
Significado de la palabra “epistemología”: muchos autores la usan para designar la “teoría del conocimiento” o “gnoseología”, sector de la filosofía que examina el problema del conocimiento en general. En este texto, “epistemología” se referirá a los problemas del conocimiento científico y los criterios con los cuales se lo justifica o invalida. La epistemología sería el estudio de las condiciones de producción y de validación del conocimiento científico.

El epistemólogo no acepta sin crítica el conocimiento científico sino que lo examina del modo más objetivo posible: frente a cualquier teoría, se preguntará por su aparición como fenómeno histórico, social o psicológico y porqué hay que considerarlo como buena o mala.

La filosofía de la ciencia abarca muchos problemas que no son estrictamente epistemológicos.

En la metodología, el metodólogo, a diferencia del epistemólogo, no pone en tela de juicio el conocimiento ya obtenido y aceptado por la comunidad científica. Su problema es la búsqueda de estrategias para incrementar el conocimiento; para esto debe usar recursos epistemológicos pues debe poseer criterios para evaluar si lo obtenido es genuino o no.

Contextos
Según Hans Reichenbach en su libro “Experiencia y predicción” establece:
Contexto de descubrimiento: solo importa la producción de una hipótesis o de una teoría, el hallazgo de una idea, la invención de un concepto, todo ello relacionado con circunstancias personales, psicológicas, sociológicas, políticas, etc. que pudiesen haber gravitado en la gestación del descubrimiento o influido en su aparición. Se relaciona con la y sociología.
Contexto de justificación: aborda cuestiones de validación, cómo saber si el descubrimiento realizado es autentico o no, si realmente se ha incrementado el conocimiento disponible. Se vincula con la teoría del conocimiento y con la lógica.
Contexto de aplicación: se discuten las aplicaciones del conocimiento científico, su utilidad, su beneficio o perjuicio para la comunidad.

En general las discusiones epistemológicas pueden llevarse a cabo en cualquiera de los tres contextos.

\end{document}
