%%%% PROCESAR con PdfLaTeX !!!!!


\documentclass[12pt]{book}
\usepackage{geometry}\geometry{top=2cm,bottom=2cm,left=3cm,right=3cm}
\usepackage{amssymb}
\usepackage{amsmath}
\usepackage{graphicx}
\usepackage{txfonts}




\begin{document}
\thispagestyle{empty}

\begin {center}

\includegraphics[scale=.4]{descarga.jpeg}

\medskip
UNIVERSIDAD DE BUENOS AIRES

Facultad de Derecho


\vspace{3cm}

\textbf{\large Teor\'ia General del Derecho}
\textbf{\large C\'atedra Portela}
\vspace{2cm}


Este es un modesto aporte para los alumnos de la f\'acultad de derecho de la UBA para la carrera de Abogac\'ia.
De ninguna man\'era pretende ser una gu\'ia de estudio, ni remplaza las clases presenciales, el material oficial de la catedra esta disponible en el web site de la m\'ateria.
\\
http://catedraportela.blogspot.com/

\end {center}


\vspace{2.5cm}

\noindent Autor:\,	Isaac Edgar Camacho Ocampo
 
\noindent Carrera:\,	Abogac\'ia

\vspace{1cm}

\vspace{1cm}

\noindent Buenos Aires, 2020

\newpage


\tableofcontents
\chapter{La etimologia del termino Derecho}
La investigacion de las palabras nos puede ser util para comprender el significado expresado por ellas, esto tiene alguna dificultad particular en la palabra \textbf{derecho}, porque tiene dos grandes troncos el castellano y el latin.
\begin{itemize}
\item Castellano Directus : participio pasivo del verbo dirigire
\item Latino ius: justicia.
\end{itemize}

\textbf{Haciendo un analisis del tronco latino}
Encontramos a dos grandes autores, Ulpiano y san isidoro de sevilla, este ultimo en (600 dc) escribio un libro sobre las etimologias y se proponia reunir todo el saber cientifico.
\\
podemos decir que las etimologias de isidoro de sevilla eran una especie de wikipedia.
ambos autores hacen derivar ius de iustitia, y eso desde el punto de vista linguistico es incorrecto porque eso implica obtener un vocablo simple (ius) de uno compuesto (iustitia) cuando deberia sere al revez.

Ahora si a iustitia le agregamos el termino STARE nos encontramos con que la justicia seria algo asi como estar en el derecho, entonces los juristas romano no temian en hablar de la justicia y derecho como sinonimos, somos lo hombres del siglo 20 los que separamos el concepto de derecho del concepto de la justicia y eso trajo muchos inconvenientes.
\\
¿podria hablarse de un derecho injusto? 
desde el punto de vista etico no es posible, pero hay otros posible origenes.
\begin{itemize}
\item iubeo : mandar
\item iuvo : ayudar
\item iungo : atar o vincular
\end{itemize}

El problema es el mismo, no podemos aceptar la derivacion de un termino monosilabico de otras palabras.

Existe otra tesis que dice que existe una relacion de lo juirdico con lo divino, y eso es porque ius se derivaria de \textbf{iupiter} es decir del dios mas importante de la mitologia romana, algunos sostienen que eso es falso porque iupoiter proviene de iupatter es decir dios padre.
De aqui se desprenden muchas palabras, iuranentum, iurare que son palabra que tenian mucho peso como una ley.
\\
Podemos concluir que cuando analisamos la etimologia del latin nos encontramos con falsas etimologias entonces buscamos fuera del latin.
\\
Nos encontramos con un sustantivos indoeuropeo, en sanscrito, nos encontramos ryeus, ryous, yowos todas estas palabras nos dan la idea de orden, entonces el ius podia ser una formula que ata o vincula ordenadamente.
\\
Podriamos suponer que el vocablo ius denota una relacion entre personas, pero no cualquier tipo de relacion sino una relacion ordenanda. 
Aparece como un adjetivo de esas palabras, iustitia, juez , jurisdiccion.

\textbf{Pasando al tronco castellano.}
\\
Directus: dirigirire, 
Reg : extenderse hacia, tambien significa regir o enderezar, tambien rex o rey.  regla recto rectum.
aparece un significado geometrico de la palablra derecho es lo opuesto a torcido.

tanto en terminos del tronco latino o germanico existe muchas diferencias, pero se advierte que hay algo  en comun, eso aparece ni bien aislamos la raiz, separando de ella los restos.

di- recto, combinando esas palabras podemos definir.

DERECHO : es una ordenacion de la convivencia humana por medio de la imposicion de unas conductas de rectitud.
\\
\\
Al dia de hoy el conjunto de los autores de derecho, no se ponen de acuerdo en los origenees del derecho, existe un autor \textit{Gustave Flaubert} y penso que podia escribir un diccionario que lo llamo el diccionario de los lugares comnunes.
\\
Flober define a derecho como: no se sabe lo que es.
\\
El termino derecho es plenamente y pluralmente significativo, ya que tiene o designa una referencia empirica, podemos hablar en muchos sentidos de experiencias juridicas, por ejemplo todo el tiempo tenemos una experiencia empirica.
\\
tomar un cafe: como contrato de locacion de servicio, viajar un contrato de transporte, este tiene una serie de implicancias porque pagando el pasaje como deber del pasajero y el chofer debe llevarnos sanso y salvos al destino si pasa algo es responsabilidad del chofer.

El derecho a la vida, el gran problema del mundo moderno es que siempre hablamos de derechos y no de los deberes ya que son dos unidades indivisibles.

Ejem. tenemos derecho a estudiar, pero tenemos deber de aprender y tambien existe derecho al trabajo, pero tambien esta el deber de cumplir con el trabajo pactado.
\\
el termino derecho tiene tambien una dimension de objetividad, porque el derecho es una realidad compleja que tiene norma , instituciones, comportamientos etc.
entonce notamos que las aguas se dividen y aparecen dos corrientes filosoficas a saber, positivismo versus iusnaturalismo ambas acerca del termino Derecho dicen que existen similitudes y deferencias.

\textbf{Positivista: }Encontramos que derecho tiene un problema de ambiguiedad, y se ve cuando vemos que el derecho tiene diversos significados.

\begin{enumerate}

\item Derecho objetivo: para el positivismo juridico es el conjuto  de normas juridicas.
\item Derecho subjetivo: es la facultads de hacer o no hacer algo, que erste autorizado por la ley.
\item  Ciencia del derecho
\item el derecho en el sentido de justicia.

\end{enumerate}


Si aceptamos las convenciones al difenerenciar los significados podemos resolver los problemas de ambiguedad del derecho.

Kelsen: en la primera mitad del siglo 20 dice que el derecho subjetivo es parte del derecho objetivo porque si alguien tienen la facultads de hacer o no hacer algo, es porque existe una norma que se lo permite o que se lo prohibe, entonces el derecho subjetivo se puede recucir a un sub conjunto del derecho objetivo.

El significado del derecho implica reducir su vaguedad.
HART: escribio un  libro de teoria del derecho, este autor renuncia a dar una definicion del termino derecho, el dice que hay que explicar derecho para las sociedades desarrolladas y no en las primitivas.
el reduce el campo de aplicacion del concotpo de derecho, asi reduce cuales son los problemas principales del derecho,

\begin{itemize}
\item Hasta que punto el derecho es soslo una cuestio de normas.
\item En que se diferenciala obligacio juridica de la obligacion moral. ¿hay diferencias y cuales son?es decir distincion de obligacion juridica y moral.
\item En que se diferencias el derecho de las ordenes con amenazas, es decir oustin de la segunda mitad del siglo 19, decia que el derecho eran ordenes repaldadas por amenazas, esto es un problema porque cual seria la diferencia entre las ordenes de un policia y las que pudiera impartirn un delincuente.
\end{itemize}

Segun Haart el derecho es un conjunto de normas primarias y secundarias.
la estrategia de haart para definir derecho es restringir la aplicacion derecho a las sociedades actuales desarrolladas y ademas el derecho es un conjunto de normas primarias y secundarias
Las primarias serian las de las sociedades primitivas es decir las que nos obligan a hacer o no hacer algo, y las secundarias son las que existen en las sociedades desarrolladas y existen tres tipos.


\begin{itemize}
\item Norma de reconocimiento: es aquella que me permite distinguir cuando estoy frente a una norma juridica valida. para haart es la norma secuandaria fundamental.
\item Norma de adjudicacion: o de juicio son aquellas que me indican que tipo de funcionarios estan autorizados para resolver los conflictos, es decir que caracteristicas deben tener esos funcionarios.
\item Norma de cambio: son quellas que me indican como debo suplir o cambiar una norma con otra norma.
¿donde encontramos normas de adjudicacion? las encontramos en la CN
¿nomras de cambio? tambien en la CN porque me da una serie de pautas para derogar una norma 
¿la regla de reconocimiento?
\end{itemize}

El derecho de una sociedad actual desarrollada viene dado por el conjunto de normas que satisfacen las condiciones impuestas por la regla de reconocimeinto incluida esa misma regla.

esta seria la explicacion del concepto del derecho para un filosofo iuspositivista.
\\
explicacion del concepto del derecho para un filosofo iusnaturalista.
existe una conisidencia, el derecho es un conpeto afectado de polisemia es decir muchos significados,
pero en gral ese conjunto de significados se reduce a 4 acepciones. 
\begin{itemize}
\item Derecho objetivo en el sentido moderno del termino, es el conjunto de normas vigentes en una comunidad determinada
\item Derecho subjetivo, en el sentido de facultad de hacer , no hacer o exigir algo. tengo derecho a..
\item Derecho expresa ideal etico de justicia, por ejemplo "no hay derecho a hacer aso" esto expresa un valor.
\item Derecho referido al saber humano aplicado al mundo juridico, el derecho como ciencia.
\end{itemize}

en castellano el termino derecho se presta aconfuciones porque no tenemos como en el ingles terminos distintos como law y rigth, solo tenemos derecho.
Como resuelve este problema la logica clasica el iusnaturalismo, lo resuelve a partis de la doctrina de la analogia, 
la logica clasica reconoce la existencia de tres clases de terminos por razon de su significacion.


\begin{itemize}
\item los terminos univocos: son aquellos que designan a una sola realidad, por ejemplo el termino hombre
\item los terminos equivocos: expresan una pluralidad de realidades entre las cuales no existe conexion, estos puede conducir a errores por ejemplo el termino bomba el explosivo o a la factura.
\item los terminos analogos: son aquellos que designan un pluralidad de realidades pero esas realidades estan conectadas, es decir que existe una conexion o analogia, ademas en los casos de analogia hay una de esas realidades designadas por el termino a la cual el termino se aplica de una manera mas conveniente. ahi estamos frente a un analogado principal y los demas son terminos analogados secundarios.
\end{itemize}

Esto fue tratado por Santo Tomas de Aquino que escribio \textbf{la Suma teologica} y en la segunda parte de este libro cuando tomas de Aquino estudia al derecho, llega a la  conclusion de que el derecho es el objeto de la justicia, y dice que es frecuente que los nombre vayan cambiando y pasen a significar cosas diferentes a lo que originalmente significarban.
por ejemplo medician antes era el remedio luego fua la ciencia medica.

lo mismo paso con el Derecho, que originalmente era \textbf{ipsa res iusta} la misma cosa justa, en segundo lugar fue el arte o la ciencia del derecho, en tercer lugar derecho se refiere al lugar donde se otorga el derecho, en cuarto lugar Derecho desgina a ala sentencia judicial, y Aquino dice que la sentencia judicial aun cuando lo que se resuelva sea inicuo (injusto), con lo cual  podemos aceptar que Aquino aceptaba la existencia de un derecho injusto.

si estudiamos estas cuatro acepciones nos daremos cuenta que en ellas no figuran las normas, tampoco las leyes ni los derechos subjetivos, hoy se discute que en el derecho medieval no existian los derechos subjetivos como los conocemos hoy.

En otro pasaje de la Suma teologica, Aquino dice \textbf{la ley no es el derecho mismo sino cierta razon del derecho} la palabra razon no esta usada como la facultad intelectual, sino como motivo o causa.

por otra parte en la expresion \textbf{ipsa res iusta} la palabra res (cosa en latin) no designa a las cosas fisicas, porque estas pueden o no participar de relaciones juridicas, es decir un conflicto juridico no siempre surge por cosas sino  tambien por conductas, entonces res designa a las acciones u omisiones humanas relativas rectamente a otros.

DERECHO: podemos decfinirlo como toda accion u omision o dacion de cosas, debida a otro con estricta de deber ser y segun una igualdad estricta respecto del titulo respecto de ese otro.

en consecuancia si nosotros hablamos de los analogados en sentido clasico, lo que deberiamos hacer es decir que el analogado principla del termino derecho es la misma cosa justa, tambien llamado derecho objetivo en el sentidfo autentico del termino, ojo porque para los positivistas el derecho objetivo era un conjunto de normas, pero para los naturalista es \textbf{res ipsa iusta }porque el derecho es el objeto de la justicia, el segundo analogado es el derecho normativo mal llamado derecho objetivo por los autores modernos, ya que si quiero referirme al derecho como un conjunto de normas entonce s logico que los llamemos derecho normativo.
en tercer luga rel derecho subjetivo porque nadie puede negar la evolucion de los derechos subjetivos.




\section{Conocimientos previos}
\section{Estado del arte}

\chapter{Conociemiento juridico}
\section{Conocimeinto}
La posibilidad de conocer es filosofia, y la parte de ella que trata este tema es la gnoseologia, el conociemeinto es una relacion entre dos parte un sujeto que conoce y un objeto que es conocido, el objeto es un ente y la parte de la filosofia que trata a los objetos en tanto entes es la ontologia.
\\
los entes pueden ser sensibles o ideales.
la relacion tiene caracteristicas es decir no spodemos preguntar como es la relacion, esto a dado lugar a muchas posturas, idealismo kantiano, realismo, esceptisismo etc.

Tipo de ralicion de conocimiento
\begin{itemize}
\item tipo de relacion social porque permite el intercmabio con otros
\item practica: el hombre quiere concocer para poder entablar relaciones con otros seres y dominar la realidad, o por disfrutar al conocer
\item dialectica: porque es un pensamiento en movimiento el sujeto puede ser objeto, en una dinamica de tesis antitesis y sintesis.
\item historica; no conocemos lo mismo que hace 100 años hay otro contexto
\end{itemize}

\\
En cuanto al objeto dijimos que la ontologia es la rama que estudia a los objetos, y tmabien existem metodologias acerca de como podemos conocerlos, empirismo, induccion, deduccion etc.

¿cuales son los tipos de conociemiento? conocimiento entendido como saber y como estado opuesto a la ignorancias.

\begin{itemize}
\item vulgar
\item Critico
\item  
\end{itemize}

minuto 7
\subsection{Definición de variables}
\subsection{Pruebas y refutaciones}
\section{Hipótesis}
\chapter{Resultados}
\section{Simulación de resultados}
\subsection{Suposiciones}
\subsection{Modelos}
\section{Resultados preliminares}
\section{Resultados postprocesados}
\subsection{Valores atípicos}
\subsection{Correlaciones}
\chapter{Conclusiones}
\end{document}
