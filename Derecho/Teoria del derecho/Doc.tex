%%%% PROCESAR con PdfLaTeX !!!!!


\documentclass[12pt]{book}
\usepackage{geometry}\geometry{top=2cm,bottom=2cm,left=3cm,right=3cm}
\usepackage{amssymb}
\usepackage{amsmath}
\usepackage{graphicx}
\usepackage{txfonts}
%\usepackage{hyperref}
\usepackage[hidelinks]{hyperref}
\hypersetup{
    colorlinks=true,
    linkcolor=blue,
    filecolor=magenta,      
    urlcolor=cyan,
    pdftitle={Sharelatex Example},
    bookmarks=true,
%    pdfpagemode=FullScreen,
}
\urlstyle{same}

\usepackage[spanish]{babel}
\setcounter{tocdepth}{3}



\begin{document}
\thispagestyle{empty}

\begin {center}

\includegraphics[scale=.4]{descarga.jpeg}

\medskip
UNIVERSIDAD DE BUENOS AIRES

Facultad de Derecho


\vspace{3cm}

\textbf{\large Teor\'ia General del Derecho}
\textbf{\large C\'atedra Portela}
\vspace{2cm}


Este es un modesto aporte para los alumnos de la f\'acultad de derecho de la UBA para la carrera de Abogac\'ia.
De ninguna man\'era pretende ser una gu\'ia de estudio, ni remplaza las clases presenciales, el material oficial de la catedra esta disponible en el web site de la m\'ateria.
\\
\url{http://catedraportela.blogspot.com/}

\end {center}


\vspace{2.5cm}

\noindent Autor:\,	Isaac Edgar Camacho Ocampo
 
\noindent Carrera:\,	Abogac\'ia

\vspace{1cm}

\vspace{1cm}

\noindent Buenos Aires, 2020
\vspace{1cm}
\\
\textit{Si se encuentra alg\'un error u omisi\'on en este res\'umen por favor colaborar en \\
\url{https://github.com/IsaacEdgarCamacho/Apuntes/tree/master/Derecho} \quad \\ o escribirme a \\ \url{isaac.edgar.camacho@gmail.com}
}
\newpage


\tableofcontents
\chapter*{Resumen} % si no queremos que añada la palabra "Capitulo"
\addcontentsline{toc}{section}{Resumen} % si queremos que aparezca en el índice
\markboth{RESUMEN}{RESUMEN} % encabezado
Este trabajo nace como un compendio del material de estudio, de la cátedra de Teoría del Derecho del Dr. Jorge Portela de la UBA.
\\
Respecto del derecho, para abordar su estudio, y como hilo conductor se plantean las siguientes interrogantes
¿Cuál es el origen del derecho?
¿existe relación entre moral y derecho? 
¿que es una Norma?
Para responder cabalmente a esas preguntas es necesario tener en cuenta a los distintos autores que trataron los temas.


\chapter{¿Que es el Derecho?}

\section{Introducción}

 
Basado en estudio recientes \cite{BERMUDEZ} se puede comprobar que... y 
complementando este hecho por lo desarrollado previamente \cite{Luckie2010} 
podemos afirmar que...
 
\section{Derecho en el pensamiento de Tomas de Aquino \cite{BERMUDEZ}}
Recordando la definición de derecho dada en las cátedras iniciales de Derecho nos encontraremos con que, tiene dos acepciones principales.
\begin{itemize}
\item el derecho como conjunto de normas, llamado “derecho objetivo”, externo al hombre y
que se le impone por coacción.
\item el derecho como potestad o facultad de obrar, llamado “derecho subjetivo” interno al
sujeto normado.
\end{itemize}
En estas definiciones no se hace referencia a la justicia, y he oído en alguna clase de otra
materia también “que como abogados no nos debe interesar el concepto de justicia porque nosotros no somos dios”.
\\
El verdadero punto de partida de la filosofía del derecho, no será la norma, la facultad, la ciencia jurídica, el lenguaje, la
sociedad o a la historia, entre otros, sino, la interrogante sobre lo justo, es decir que es los justo.

La investigacion de las palabras nos puede ser util para comprender el significado expresado por ellas, esto tiene alguna dificultad particular en la palabra \textbf{derecho}, porque tiene dos grandes troncos el castellano y el latin.
\begin{itemize}
\item Castellano Directus : participio pasivo del verbo dirigire
\item Latino ius: justicia.
\end{itemize}

\textbf{Haciendo un analisis del tronco latino}
Encontramos a dos grandes autores, Ulpiano y san isidoro de sevilla, este ultimo en (600 dc) escribio un libro sobre las etimologias y se proponia reunir todo el saber cientifico.
\\
podemos decir que las etimologias de isidoro de sevilla eran una especie de wikipedia.
ambos autores hacen derivar ius de iustitia, y eso desde el punto de vista linguistico es incorrecto porque eso implica obtener un vocablo simple (ius) de uno compuesto (iustitia) cuando deberia sere al revez.

Ahora si a iustitia le agregamos el termino STARE nos encontramos con que la justicia seria algo asi como estar en el derecho, entonces los juristas romano no temian en hablar de la justicia y derecho como sinonimos, somos lo hombres del siglo 20 los que separamos el concepto de derecho del concepto de la justicia y eso trajo muchos inconvenientes.
\\
¿podria hablarse de un derecho injusto? 
desde el punto de vista etico no es posible, pero hay otros posible origenes.
\begin{itemize}
\item iubeo : mandar
\item iuvo : ayudar
\item iungo : atar o vincular
\end{itemize}

El problema es el mismo, no podemos aceptar la derivacion de un termino monosilabico de otras palabras.

Existe otra tesis que dice que existe una relacion de lo juirdico con lo divino, y eso es porque ius se derivaria de \textbf{iupiter} es decir del dios mas importante de la mitologia romana, algunos sostienen que eso es falso porque iupoiter proviene de iupatter es decir dios padre.
De aqui se desprenden muchas palabras, iuranentum, iurare que son palabra que tenian mucho peso como una ley.
\\
Podemos concluir que cuando analisamos la etimologia del latin nos encontramos con falsas etimologias entonces buscamos fuera del latin.
\\
Nos encontramos con un sustantivos indoeuropeo, en sanscrito, nos encontramos ryeus, ryous, yowos todas estas palabras nos dan la idea de orden, entonces el ius podia ser una formula que ata o vincula ordenadamente.
\\
Podriamos suponer que el vocablo ius denota una relacion entre personas, pero no cualquier tipo de relacion sino una relacion ordenanda. 
Aparece como un adjetivo de esas palabras, iustitia, juez , jurisdiccion.

\textbf{Pasando al tronco castellano.}
\\
Directus: dirigirire, 
Reg : extenderse hacia, tambien significa regir o enderezar, tambien rex o rey.  regla recto rectum.
aparece un significado geometrico de la palablra derecho es lo opuesto a torcido.

tanto en terminos del tronco latino o germanico existe muchas diferencias, pero se advierte que hay algo  en comun, eso aparece ni bien aislamos la raiz, separando de ella los restos.

di- recto, combinando esas palabras podemos definir.

DERECHO : es una ordenacion de la convivencia humana por medio de la imposicion de unas conductas de rectitud.
\\
\\
Al dia de hoy el conjunto de los autores de derecho, no se ponen de acuerdo en los origenees del derecho, existe un autor \textit{Gustave Flaubert} y penso que podia escribir un diccionario que lo llamo el diccionario de los lugares comnunes.
\\
Flober define a derecho como: no se sabe lo que es.
\\
El termino derecho es plenamente y pluralmente significativo, ya que tiene o designa una referencia empirica, podemos hablar en muchos sentidos de experiencias juridicas, por ejemplo todo el tiempo tenemos una experiencia empirica.
\\
tomar un cafe: como contrato de locacion de servicio, viajar un contrato de transporte, este tiene una serie de implicancias porque pagando el pasaje como deber del pasajero y el chofer debe llevarnos sanso y salvos al destino si pasa algo es responsabilidad del chofer.

El derecho a la vida, el gran problema del mundo moderno es que siempre hablamos de derechos y no de los deberes ya que son dos unidades indivisibles.

Ejem. tenemos derecho a estudiar, pero tenemos deber de aprender y tambien existe derecho al trabajo, pero tambien esta el deber de cumplir con el trabajo pactado.
\\
el termino derecho tiene tambien una dimension de objetividad, porque el derecho es una realidad compleja que tiene norma , instituciones, comportamientos etc.
entonce notamos que las aguas se dividen y aparecen dos corrientes filosoficas a saber, positivismo versus iusnaturalismo ambas acerca del termino Derecho dicen que existen similitudes y deferencias.

\textbf{Positivista: }Encontramos que derecho tiene un problema de ambiguiedad, y se ve cuando vemos que el derecho tiene diversos significados.

\begin{enumerate}

\item Derecho objetivo: para el positivismo juridico es el conjuto  de normas juridicas.
\item Derecho subjetivo: es la facultads de hacer o no hacer algo, que erste autorizado por la ley.
\item  Ciencia del derecho
\item el derecho en el sentido de justicia.

\end{enumerate}


Si aceptamos las convenciones al difenerenciar los significados podemos resolver los problemas de ambiguedad del derecho.

Kelsen: en la primera mitad del siglo 20 dice que el derecho subjetivo es parte del derecho objetivo porque si alguien tienen la facultads de hacer o no hacer algo, es porque existe una norma que se lo permite o que se lo prohibe, entonces el derecho subjetivo se puede recucir a un sub conjunto del derecho objetivo.

El significado del derecho implica reducir su vaguedad.
HART: escribio un  libro de teoria del derecho, este autor renuncia a dar una definicion del termino derecho, el dice que hay que explicar derecho para las sociedades desarrolladas y no en las primitivas.
el reduce el campo de aplicacion del concotpo de derecho, asi reduce cuales son los problemas principales del derecho,

\begin{itemize}
\item Hasta que punto el derecho es soslo una cuestio de normas.
\item En que se diferenciala obligacio juridica de la obligacion moral. ¿hay diferencias y cuales son?es decir distincion de obligacion juridica y moral.
\item En que se diferencias el derecho de las ordenes con amenazas, es decir oustin de la segunda mitad del siglo 19, decia que el derecho eran ordenes repaldadas por amenazas, esto es un problema porque cual seria la diferencia entre las ordenes de un policia y las que pudiera impartirn un delincuente.
\end{itemize}

Segun Haart el derecho es un conjunto de normas primarias y secundarias.
la estrategia de haart para definir derecho es restringir la aplicacion derecho a las sociedades actuales desarrolladas y ademas el derecho es un conjunto de normas primarias y secundarias
Las primarias serian las de las sociedades primitivas es decir las que nos obligan a hacer o no hacer algo, y las secundarias son las que existen en las sociedades desarrolladas y existen tres tipos.


\begin{itemize}
\item Norma de reconocimiento: es aquella que me permite distinguir cuando estoy frente a una norma juridica valida. para haart es la norma secuandaria fundamental.
\item Norma de adjudicacion: o de juicio son aquellas que me indican que tipo de funcionarios estan autorizados para resolver los conflictos, es decir que caracteristicas deben tener esos funcionarios.
\item Norma de cambio: son quellas que me indican como debo suplir o cambiar una norma con otra norma.
¿donde encontramos normas de adjudicacion? las encontramos en la CN
¿nomras de cambio? tambien en la CN porque me da una serie de pautas para derogar una norma 
¿la regla de reconocimiento?
\end{itemize}

El derecho de una sociedad actual desarrollada viene dado por el conjunto de normas que satisfacen las condiciones impuestas por la regla de reconocimeinto incluida esa misma regla.

esta seria la explicacion del concepto del derecho para un filosofo iuspositivista.
\\
explicacion del concepto del derecho para un filosofo iusnaturalista.
existe una conisidencia, el derecho es un conpeto afectado de polisemia es decir muchos significados,
pero en gral ese conjunto de significados se reduce a 4 acepciones. 
\begin{itemize}
\item Derecho objetivo en el sentido moderno del termino, es el conjunto de normas vigentes en una comunidad determinada
\item Derecho subjetivo, en el sentido de facultad de hacer , no hacer o exigir algo. tengo derecho a..
\item Derecho expresa ideal etico de justicia, por ejemplo "no hay derecho a hacer aso" esto expresa un valor.
\item Derecho referido al saber humano aplicado al mundo juridico, el derecho como ciencia.
\end{itemize}

en castellano el termino derecho se presta aconfuciones porque no tenemos como en el ingles terminos distintos como law y rigth, solo tenemos derecho.
Como resuelve este problema la logica clasica el iusnaturalismo, lo resuelve a partis de la doctrina de la analogia, 
la logica clasica reconoce la existencia de tres clases de terminos por razon de su significacion.


\begin{itemize}
\item los terminos univocos: son aquellos que designan a una sola realidad, por ejemplo el termino hombre
\item los terminos equivocos: expresan una pluralidad de realidades entre las cuales no existe conexion, estos puede conducir a errores por ejemplo el termino bomba el explosivo o a la factura.
\item los terminos analogos: son aquellos que designan un pluralidad de realidades pero esas realidades estan conectadas, es decir que existe una conexion o analogia, ademas en los casos de analogia hay una de esas realidades designadas por el termino a la cual el termino se aplica de una manera mas conveniente. ahi estamos frente a un analogado principal y los demas son terminos analogados secundarios.
\end{itemize}

Esto fue tratado por Santo Tomas de Aquino que escribio \textbf{la Suma teologica} y en la segunda parte de este libro cuando tomas de Aquino estudia al derecho, llega a la  conclusion de que el derecho es el objeto de la justicia, y dice que es frecuente que los nombre vayan cambiando y pasen a significar cosas diferentes a lo que originalmente significarban.
por ejemplo medician antes era el remedio luego fua la ciencia medica.

lo mismo paso con el Derecho, que originalmente era \textbf{ipsa res iusta} la misma cosa justa, en segundo lugar fue el arte o la ciencia del derecho, en tercer lugar derecho se refiere al lugar donde se otorga el derecho, en cuarto lugar Derecho desgina a ala sentencia judicial, y Aquino dice que la sentencia judicial aun cuando lo que se resuelva sea inicuo (injusto), con lo cual  podemos aceptar que Aquino aceptaba la existencia de un derecho injusto.

si estudiamos estas cuatro acepciones nos daremos cuenta que en ellas no figuran las normas, tampoco las leyes ni los derechos subjetivos, hoy se discute que en el derecho medieval no existian los derechos subjetivos como los conocemos hoy.

En otro pasaje de la Suma teologica, Aquino dice \textbf{la ley no es el derecho mismo sino cierta razon del derecho} la palabra razon no esta usada como la facultad intelectual, sino como motivo o causa.

por otra parte en la expresion \textbf{ipsa res iusta} la palabra res (cosa en latin) no designa a las cosas fisicas, porque estas pueden o no participar de relaciones juridicas, es decir un conflicto juridico no siempre surge por cosas sino  tambien por conductas, entonces res designa a las acciones u omisiones humanas relativas rectamente a otros.

DERECHO: podemos decfinirlo como toda accion u omision o dacion de cosas, debida a otro con estricta de deber ser y segun una igualdad estricta respecto del titulo respecto de ese otro.

en consecuancia si nosotros hablamos de los analogados en sentido clasico, lo que deberiamos hacer es decir que el analogado principla del termino derecho es la misma cosa justa, tambien llamado derecho objetivo en el sentidfo autentico del termino, ojo porque para los positivistas el derecho objetivo era un conjunto de normas, pero para los naturalista es \textbf{res ipsa iusta }porque el derecho es el objeto de la justicia, el segundo analogado es el derecho normativo mal llamado derecho objetivo por los autores modernos, ya que si quiero referirme al derecho como un conjunto de normas entonce s logico que los llamemos derecho normativo.
en tercer luga rel derecho subjetivo porque nadie puede negar la evolucion de los derechos subjetivos.


\begin{thebibliography}{0}
  \bibitem{BERMUDEZ} EL CONCEPTO TOMISTA DEL DERECHO en la interpretación de Juan Alfredo Casaubon. FERNANDO ADRIÁN BERMÚDEZ
  						Biblioteca Digital de la Universidad Católica Argentina
  \bibitem{Luckie2010} Matthew Luckie. CScamper: a scalable, extensible packet 
                              prober for active measurement of the internet, 2010.
\end{thebibliography}


\chapter{Conociemiento juridico}
\section{Conocimeinto}
La posibilidad de conocer es filosofia, y la parte de ella que trata este tema es la gnoseologia, el conociemeinto es una relacion entre dos parte un sujeto que conoce y un objeto que es conocido, el objeto es un ente y la parte de la filosofia que trata a los objetos en tanto entes es la ontologia.
\\
los entes pueden ser sensibles o ideales.
la relacion tiene caracteristicas es decir no spodemos preguntar como es la relacion, esto a dado lugar a muchas posturas, idealismo kantiano, realismo, esceptisismo etc.

Tipo de ralicion de conocimiento
\begin{itemize}
\item tipo de relacion social porque permite el intercmabio con otros
\item practica: el hombre quiere concocer para poder entablar relaciones con otros seres y dominar la realidad, o por disfrutar al conocer
\item dialectica: porque es un pensamiento en movimiento el sujeto puede ser objeto, en una dinamica de tesis antitesis y sintesis.
\item historica; no conocemos lo mismo que hace 100 años hay otro contexto
\end{itemize}

En cuanto al objeto dijimos que la ontologia es la rama que estudia a los objetos, y tmabien existem metodologias acerca de como podemos conocerlos, empirismo, induccion, deduccion etc.

¿cuales son los tipos de conociemiento? conocimiento entendido como saber y como estado opuesto a la ignorancias.

\begin{itemize}
\item vulgar
\item Critico
\item  
\end{itemize}

minuto 7

\section{Derecho y Moral}
¿Existe relacion o no? Podemos nombrar a Kant como el pensador que inicia, hacia fines del siglo XVIII, la cuestión referida a lo que el
considera como una separación tajante entre el mundo de lo jurídico y el mundo de la ética.\\
Kant decia

\chapter{Caracterirstica de la norma juridica en el pensamiento Clasico}
En el diccionario de la Real Academia Española, veremos que norma tiene que ver con una “directriz” y con una “orden”, y también con la palabra “regla” que proviene del latín, “regula”.
En griego nomos, refiere a una manera de medir, aplicado al mundo jurídico, la norma jurídica
designa una directriz que se emplea para regir y enmarcar las conductas entre humanos siempre y cuando esta tenga que ver con el derecho.
\section{Introduccion}
Atravez de sus mas grandes pensadores, la sabiduria antigua concibio al derecho positivo como un reflejo de un orden superior.
Desde Aristoteles que en su \textbf{Etica a nicomaco} dice.
\\
\\
\textit{El derecho politico es una parte natural y otra legal, es natural lo que en todas partes es independiente de las opiniones y es legal lo que a priori puede ser de un modo o del contrario, sin embargo esta dicotomia cesa cuando la ley rectamente lo resuelve.}

Esto fue copiado por los romanos y definieron que el derecho es una parte natural y otra legal.
\begin{itemize}
\item Natural: es la parte del derecho que le es comun a todos los pueblos, es decir son las normas que son validas a todos los hombres sin importar costumbres ni etnias, eso es el derecho natural o derecho de gentes (ius gentium).
\item Civil: El derecho que cada pueblo se da a si mismo, es decir el conjunto de normas que cada pueblo desarrolla para si, se llama derecho civil o (ius civile).
\end{itemize}

En Resumen todo derecho que rige a un pueblo politicamente soberano e independiente, es en parte derecho natural o de origen natural y en parte derecho civil o de origen humano.
Ciceron decia que para distinguir un aley buena de una mala bastaba con ver si se ajustaba a las reglas de la naturaleza.


\subsection{Justicia prodencia, inteligencia y voluntad}
\textbf{Prudencia} Capacidad de pensar, ante ciertos acontecimientos o actividades, sobre los riesgos posibles que estos conllevan, y adecuar o modificar la conducta para no recibir o producir perjuicios innecesarios o Virtud cardinal del catolicismo que consiste en discernir y distinguir lo que está bien de lo que está mal y actuar en consecuencia.
\\
\textbf{justicia:} Principio moral que inclina a obrar y juzgar respetando la verdad y dando a cada uno lo que le corresponde.
\\
\textbf{Voluntad:} La voluntad es la aptitud de decidir y ordenar la propia conducta. Propiedad que se expresa de forma consciente en el ser humano, para realizar algo con intención de un resultado.
Podemos definirla como una terna de elementos \textbf{Discernimiento, intencion y libertad}
\begin{itemize}
\item Discernimiento o prudencia: es la aptitud de una persona de distinguir lo bueno de lo malo, lo inconveniente de lo conveniente,
\item Intencion: un acto determinado fue intencional, cuando el resultado del mismo coincide con el objetivo buecado.
\item Libertad: es la espontaneidad en la determinacion del sujeto, es decir cuando no exista coaccion.
\end{itemize}


Esta en manos del legislador determinar el orden que distinguimos antes, entre derecho natural y civil, esto lo hace utilizando una virtud cardinal como la prudencia, la cual permite reconocer la reaidad y por lo tanto obrar con justicia.
Porque ciertamente los derecho o casas que debemos distribuir son entes sensibles es decir pertenecen a la realidad de los sentidos, y si no conocemos lo real no podremos distribuir con justicia.
\\
Siendo la justicia una virtud que debe enderezar la voluntad
\subsection{Modelos}
\section{Resultados preliminares}
\section{Resultados postprocesados}
\subsection{Valores atípicos}
\subsection{Correlaciones}
\chapter{Derecho natural}
Vamos a precisar algunos detalles acerca del concepto de derecho natural y tambien otro concepto que le está muy unido que es el de naturaleza.
\\
Hablar de derecho natural implica remontarse a Platón, Aristóteles a San Agustín y Santo Tomás y a la segunda escolástica del barroco es decir a los clásicos, todo esto sin embargo parece arcaico pero vamos a notar que nunca estuvo más actualizado.
\\ \\
Para entrar de lleno al asunto tenemos que desentrañar un término que es el de \textbf{naturaleza} ¿que es la naturaleza?, Aristóteles la definió en su metafísica como: \\

\textit{Principio de vida y de movimiento de todas las cosas existentes, o como la sustancia de las cosas que tienen el principio del movimiento en sí mismas}
\\ \\
Eso también nos muestran que la naturaleza no hace nada con mezquindad es decir todo tiene un fin u objetivo, pero si la naturaleza de algo es el fin es decir la naturaleza de un ente es el fin con el que vino a este mundo , podemos ligar ese concepto el concepto de bien ya que en todas las ciencias y artes el fin u objetivo es el bien, podemos llegar a concluir que lo que se persigue es la perfección del ser humano.
\\ \\
Cuando hablamos de derecho natural fundamos al término naturaleza no en cualquiera es decir no hablamos de la naturaleza de los perros o de los osos o de cualquier otro ente, sino que estamos hablando de la naturaleza humana algo en virtud de lo cual a un ente se lo puede llamar hombre, existen dos rasgos particulares la \textbf{sociabilidad} y la \textbf{racionalidad} eso es lo que distingue la naturaleza humana de la naturaleza de cualquier otro animal viviente.
\\ \\
La naturaleza del hombre es la sustancia que permanece en el fondo de todos y cada uno de nosotros es una realidad es decir, existe,  y nosotros lo único que tendremos que hacer es descubrir esa realidad.
\\
En contrapunto con lo que estamos viviendo actualmente la naturaleza humana es algo que aparentemente se puede modificar, se nos dice que la naturaleza humana puede ser moldeable es decir se puede llegar a deformar pero existen leyes que no podemos romper, es decir no podemos convertirnos en sapos por ejemplo. \\
Pero tampoco sería deseable transformarnos en una entidad como sapos o vacas ya que eso no nos llevaría a la perfección esto es de sentido común.
\\ \\
Podemos decir entonces que existe la naturaleza humana un denominador común en todos los hombres y de ello se desprende que existan actos que podemos llamar humanos y otros que podemos llamar inhumanos como la tortura. \\
Los actos humanos son naturales al hombre es decir están en concordancia con la naturaleza ontológica del hombre y los segundos son una degeneración de la naturaleza del hombre.
\\ \\
Pero parece ser qué hay leyes que el hombre aunque quiera no puede romper y también podemos decir que las palabras inhumano o antinatural sólo tienen sentido habiendo aceptado el concepto de naturaleza humana.
\\
Por lo tanto \textbf{hay naturaleza humana}, hay un contenido, fijo, estable y común que impone unos límites a la acción y unas exigencias básicas para todos, para ricos y para pobres para gobernadores y súbditos.
\\
También se nota en la actualidad que todo sería mucho mejor si el hombre pudiera abstraerse de todo tipo de normas es decir que no haya ninguna norma qué condiciones su libertad. \\
Quien no soñ\'o alguna vez en ser libres como los pájaros, que no haya nada que nos limite pero sin embargo surge una pregunta interesante \textbf{¿son de verdad libre los pájaros?}, 
es decir podemos preguntarnos si su vuelo no está atado a leyes físicas que no puede romper, en la naturaleza hay leyes que no podemos romper, la ley de gravedad, las leyes de descendencia es decir los genes que se derivan de padre a hijo y muchas otras que aunque no queramos no podemos romper.
\\ \\
Derecho positivo y derecho natural, podemos decir que el derecho natural contiene al derecho positivo es decir que el derecho natural posibilita la existencia de un derecho positivo el derecho natural sería como la estructura sobre la que se apoya el derecho positivo son las reglas que nos dicen como hacer o construir un derecho positivo.
\subsection{Milley}
Según Aristóteles nosotros extraemos ante todo \textbf{lo justo} de la observación de la naturaleza es decir las leyes constituidas según la naturaleza.
\\
Nosotros buscamos en el mundo exterior es decir nos oponemos al método que pretende deducir la justicia a partir de la razón interna humana, aquel método supone que la naturaleza exterior nos da un orden es decir una justicia.
\\
Cómo abogados, tenemos muchas veces que mediar entre situaciones de conflicto entre dos partes, cómo podremos resolver el conflicto si buscamos en la razón interna de cada una de estas, está claro que allí encontraremos solamente argumentos que refuerza su posici\'on, es decir yo tengo la raz\'on y no el otro, pero si nos ponemos en la vereda de enfrente lo mismo puede ocurrir, es por eso que la observación de la naturaleza es trascendental para repartir lo justo a cada uno.
\\ \\

\textbf{La naturaleza es en definitiva, el modo en que cada realidad se manifiesta siendo.}
\\ \\
Si buscamos naturaleza en el diccionario este nos dice que es lo que pertenece a la naturaleza del ser lo que le es natural al ser, o la naturaleza de un ser es lo que determina las operaciones propias del ser
podemos definir entonces:
\\
\textbf{derecho natural: como la parte del orden jurídico que proviene de la naturaleza del hombre}
\\ \\
El fundamento del derecho entonces es el hombre estrictamente hablando en general es decir la esencia hombre, esta esencia es un universal porque no existe el hombre allí donde no se dé el conjunto de características que definen al hombre.
\\
Aquí entendemos a la esencia en cuanto a principios de operación es decir el conjunto de operaciones naturales al hombre.
\\ \\
Una pregunta interesante surge \textbf{¿que contiene el derecho natural?} es decir que hay dentro de lo que llamamos derecho natural.\\ \\
\textit{Debemos dilucidar el concepto de ley, Santo Tomás de Aquino dice que una ley es una orden de la razón que se dirige hacia el bien común y está promulgada por quien tiene a su cargo el cuidado de la comunidad.
}
\\ \\
existe una estructura piramidal de leyes que no es la famosa pirámide de kelsen en este estructura tenemos como cúspide de la pirámide a la ley eterna es decir la ley de Dioses decir es un principio ordenador de la universalidad de lo creado.

a esa ley eterna nosotros la conocemos a través de la ley natural es decir la ley natural nos muestran aspecto cognoscitivo de la ley eternas es decir a la ley eterna cuando el hombre la conoce se transforma en ley natural y a la ley eterna podemos acceder a través de la razón humana

y la razón humana tiene dos aspectos un aspecto teórico y uno práctico
en ambos tipos de razón tenemos dos principios que que mueven el razonamiento el primero es el principio de no contradicción es decir una entidad no puede ser y no ser al mismo tiempo

la razón práctica tiene un principio que dice que el bien debe hacerse y el mal debe evitarse.





Sindéresis


Frutos del derecho natural clásico

Existe una estructura de las leyes que enunciamos anteriormente esto es la eterna ley natural y ley positiva esta última es la ley de la ciudad la que nos rige en tanto estemos en la polis.

la ley eterna no resulta cognoscible para el hombre en tanto la ley natural es la ley eterna ya conocida por el hombre es decir es el aspecto cognoscitivo de la ley eterna.

el entendimiento humano está dividida en dos el entendimiento teórico y el entendimiento práctico o dicho de otra forma el conocimiento teórico y el conocimiento práctico y en ambos tipos de conocimientos existe un principio en el principio de conocimiento teórico podemos nombrar el principio de no contradicción en tanto en el conocimiento práctico tenemos otro principio que funciona como una orden o imperativo debe hacerse el bien y debe evitarse el mal

Hasta altura surge una pregunta interesante cómo es que el hombre llega a conocer el primer principio del conocimiento práctico tenemos que definir pues el concepto de sindéresis


Según el diccionario la sindéresis o razón natural es la apertura cognoscitiva de la persona humana a su propia naturaleza es una propuesta vigente de Tomás de Aquino también se la podría definir como la capacidad o disposición natural de la razón práctica para aprender inclusivamente los principios primeros universales de la acción humana.

En resumidas palabras podemos decir que la sindéresis es la conciencia moral que tiene cada uno de nosotros pero podemos preguntarnos cómo es que tenemos esta sindéresis la tenemos a través de un razonamiento es decir razonamos y nos damos cuenta que tenemos conciencia moral pues no la sindéresis es un tipo de conocimiento que viene impresa en la misma naturaleza del hombre por lo tanto el hombre tiene un conocimiento inmediato de los principios del orden práctico

por lo tanto podemos definir a la síntesis como una facultad innata del conocimiento humano

el profesor pone el ejemplo de por ejemplo cortarle el cuello nuestra madre para sacarle \$500 evidentemente cualquier persona puede darse cuenta de que esto es incorrecto y por lo tanto no hace falta exhaustivos razonamientos para concluir que esto es malo

a los primeros principios no hay que demostrar los sino hay que mostrar los para que todos se den cuenta de esta evidencia existe una relación entre la sindéresis y la conciencia humana de allí se derivan muchos conceptos importantes en el derecho por ejemplo la objeción de conciencia

podríamos decir también que el juicio de la sindéresis es siempre correcto porque es ese conocimiento inmediato del ser humano DS primer principio

kelsen decía en su teoría pura del derecho que no importa el contenido de la norma en tanto ésta sea vigente y que la pirámide jurídica no tiene nada que ver con la pirámide del derecho natural en donde existe un derecho externo una ley eterna en la cúspide una ley natural y una ley positiva la ley de eterna es el plan de Dios la ley natural es la ley interna conocida por el hombre y luego el concepto de sindéresis y después vienen las leyes positivas


Los frutos del derecho natural clásico

El primer fruto que podemos nombrar está referido al derecho internacional Francisco de vitoria es el autor que es reconocido como el primero que habló sobre derecho internacional enumero unos principios que son aplicables a todos los pueblos del mundo y cómo llegó a esto a través de lo que nosotros conocemos como la leyenda negra

Cuando nos hablan de la conquista de América generalmente tenemos la idea de que los españoles cometieron cualquier tipo de abuso es decir se apropian de las pertenencias de los nativos existían violaciones a las mujeres y todo tipo de aberraciones sin embargo cuando analizamos la historia verdadera nos encontramos con que en los viajes que realizaban los españoles hacia América existían sacerdotes y cuando se comete algún tipo de abuso está sacerdotes de acuerdo a su voto emitían una denuncia hacia los Reyes católicos y por supuesto existieron muchos abusos pero lo que no se dice son las medidas tomadas por los Reyes católicos a partir de ésta denuncia

católicos al recibir esta denuncias elevaron todo esto a la universidad de Salamanca que era la universidad más prestigiosa de España para que emita un dictamen acerca de la pertenencia de los nativos al género humano es decir había que decir si los nativos eran o no personas porque de no serlo existía la posibilidad de que se pudiera hacer con ellos lo que venga en gana entonces Francisco de vitoria ayudado por Fray Bartolomé de las casas llegan a la conclusión de que los indígenas son seres humanos aunque no estuvieran bautizados

A ese conjunto de dictámenes que metió la corona española se le llamó legislación de indias para para los territorios conquistados que era una prolongación de la corona española y no una colonia como su análogo inglés por eso se tiene Francisco de vitoria como fundador del derecho internacional público

Otro fruto del derecho natural son en los hechos ocurridos durante Las guerras mundiales tras los hechos acaecidos en la década del 30 y 40 principalmente en Europa pero en todo el resto del mundo el derecho positivo sea conmovido al no poder dar respuesta y esto en 1948 dio origen a la declaración universal de los derechos humanos en el artículo 38 dice claramente que se reconocen los derechos a todas las personas en todas las sociedades del mundo

Otro fruto del derecho natural es la justicia social generalmente asociamos este concepto en Argentina al peronismo pensando en que el general perón fue el primero en hablar del tema pero esto no es así ya que podemos ver en el siglo 19 en la doctrina de los pontífices cuando se reconoció a la cuestión social
Podemos citar muchas encíclicas en la que se nombra la importancia de la cuestión social de la desventaja que cuenta el obrero respecto del patrón de la importancia que tienen los contratos de trabajo entonces la justicia social hace que el bien común prevalezca que el bien común sea participable es un bien del cual puede nutrirse todos los integrantes del cuerpo social

Tercer fruto concreto el principio de subsidiariedad y los cuerpos intermedios

Existen un conjunto de entidades que están entre la familia que es la primer célula donde el hombre nace y se desarrolla y el estado que es la sociedad política estas entidades son por ejemplo la escuela los clubes de Barrio Las bibliotecas los consejos profesionales las universidades es decir entidades en donde el hombre progresa se cultiva y avanza esta es una creación doctrinaria del derecho natural
Porque subsidiariedad porque toda actividad social es por esencia subsidiaria es decir sirve de apoyo a los miembros de la sociedad sin jamás absorberlos o destruir los

Otro fruto concreto del derecho natural hermenéutica jurídica
En cuanto a la interpretación jurídica dijimos que existen muchos problemas y el gran aporte que da el derecho natural es la doctrina de la equidad que según Aristóteles es la dichosa rectificación de lo justo rigurosamente legal

Por eso la equidad es la justicia del caso particular es tu tele no decía que la equidad se asemeja a la regla lesvia que erauna herramienta que usaban los arquitectos de la isla de lesbos para medir piedras y esta era flexible y se podía adaptar a la forma de cualquier piedra la piedra es el caso que tenemos que solucionar con un criterio de justicia



Por último nos vamos a referir al derecho a la vida
porque la defensa del derecho a la vida es una constante en el derecho natural clásico porque simplemente con la vida se cumple el destino del hombre es decir no hay ningún destino si no hay vida antes
A partir de la vida surgen los derechos naturales más concretos
El derecho a la vida implica el derecho a no ser víctima de un homicidio por lo tanto referido al aborto podemos decir que es un homicidio calificado porque porque generalmente lo hacen los padres y podemos decir que el nonato no es defendido por nadie

El derecho a la vida también implica el derecho de toda persona adulta utilizar para su propia perfección sus facultades físicas naturales
también el derecho a la vida implica el derecho a conservar la vida una vez que la tenemos
escritorio también implica un derecho a conservar todas las partes de nuestro cuerpo incluidos aquellos que no son vitales








\end{document}
