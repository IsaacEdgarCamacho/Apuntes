\documentclass[a4paper,16pt]{article}

%%%%%%%%%%%% PREÁMBULO %%%%%%%%%%%%%%%%%%%%%

% Paquetes

\usepackage[utf8]{inputenc}
\usepackage[spanish,es-tabla]{babel}
\usepackage[T1]{fontenc}

\usepackage{listings}

% Comandos

\renewcommand{\lstlistingname}{Código}
\renewcommand{\lstlistlistingname}{Índice de fragmentos de código fuente}

% Opciones

\title{COMBINATORIA}
\author{Isaac Edgar Camacho}

%%%%%%%%%%%%%%%%%%%%%%%%%%%%%%%%%%%%%%%%%%%%%%
%margenes

\setlength{\textwidth}{120mm}
\setlength{\textheight}{200mm}



\begin{document}
\maketitle

\begin{abstract}
Cuando ingresamos al mundo universitario, y nos enfrentamos a cursos de Algebra o C\'alculo entre otras, nos damos cuenta que estamos en serios problemas porque queremos entender conceptos complejos por su nivel de abstracci\'on y no nos damos cuenta que ni siquiera sabemos contar!!! 
\end{abstract}


\section{APRENDIENDO A CONTAR}

\begin{itemize}
\item \textbf{Regla de la suma}\\   
Supongamos que podemos ir de un lugar X a otro Y, ademas supongamos que podemos ir caminando o en taxy o en colectivo \\¿De cuantas formas puedo llegar de X a Y? \\ la respuesta por la relga de la suma es de 3 maneras diferentes!!
\begin{center}
\textbf{
Si una primer tarea puede realizarse de \textit{m} formas, mientras que otra tarea puede realizarse de \textit{n} formas, entonces si no se pueden hacer ambas a la vez, cualquiera de ellas pueden realizarse de \textit{m + n} formas.}
\[
\sum_{i=1}^{r}x_{i}=x_{1}+x_{2}+x_{3} 
\]

\end{center}

\item \textbf{Regla del producto}\\  
Supongamos que podemos ir de un lugar X a otro Y por ultimo a Z, ademas supongamos que podemos ir de X a Y caminando o en taxy o en colectivo, pero ademas desde Y hasta Z podemos ir en tren o a caballo\\¿De cuantas formas puedo llegar de X a Z? \\ la respuesta por la relga del producto es de 3 x 2 = 6 maneras diferentes!!
\begin{center}
\textbf{
Si un proceso se puede descomponer en dos etapas y el primero tiene \textit{m} resultados posibles, mientras que el segundo tiene \textit{n} resultados posibles, entonces el procedimiento total puede realizarse de \textit{m x n} formas.}

\[
\prod_{i=1}^{3}x_{i}=x_{1}\cdot x_{2}\cdot x_{3} 
\]
\end{center}

\item \textbf{Permutaciones}\\  
Siguiendo con las aplicaciones de la regla del producto, ahora contamos disposiciones lineales de objetos.
\begin{center}
\textbf{
Si un proceso se puede descomponer en dos etapas y el primero tiene \textit{m} resultados posibles, mientras que el segundo tiene \textit{n} resultados posibles, entonces el procedimiento total puede realizarse de \textit{m x n} formas.}
\end{center}




\item Garbage colector: quita los objetos a los que no haga referencia nada
\end{itemize}

\end{document}