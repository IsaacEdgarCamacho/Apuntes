\documentclass[a4paper,12pt]{article}

%%%%%%%%%%%% PREÁMBULO %%%%%%%%%%%%%%%%%%%%%

% Paquetes

\usepackage[utf8]{inputenc}
\usepackage[spanish,es-tabla]{babel}
\usepackage[T1]{fontenc}

\usepackage{listings}

% Comandos

\renewcommand{\lstlistingname}{Código}
\renewcommand{\lstlistlistingname}{Índice de fragmentos de código fuente}

% Opciones

\title{Analisis Matem\'atico 2 fiuba}
\author{Isaac Edgar Camacho}

%%%%%%%%%%%%%%%%%%%%%%%%%%%%%%%%%%%%%%%%%%%%%%
%margenes

\setlength{\textwidth}{120mm}
\setlength{\textheight}{200mm}



\begin{document}
\maketitle

\begin{abstract}
Cuando uno se enfrenta a un primer curso universitario de c/'alculo siempre debe lidiar con conocimiento previo que debi/'o haber adquirido en la educaci\'on secuandaria, pero en general nunca vienen preparados para entender conceptos complejos.

\end{abstract}

\tableofcontents

\section{Puntos en el Espacio}

\begin{itemize}
    \item En $R^{2}$ un punto se indica como un par ordenado x = (x,y) es decir	un par de coordenadas cartecianas y vamos a decir que x pertenece a $R^{2}$ si y solo si x  tiene dos componentes.
		De la misma manera un punto en $R^{3}$ se escribe como una terna ordenada x = (x,y,z) y 				decimos que x pertenece a $R^{3}$ si y solo si x tiene tres componentes.
		
		Podemos generalizarlo a $R^{n}$, en este espacio un punto se indica como una n-upla 
		x = $(\,\underbrace{a,\ldots, a}_{n}\,)$ entonces decimos que x pertenece a $R^{n}$ si y solo si x tiene n componentes.
\\		El espacio $R^{2}$ Algebraicamente es el conjunto de todos los pares ordenados y 						Geometricamente se lo puede definir como el conjunto de todos los vectores que parten 		del origen.

    \item \textbf{ESPACIOS VECTORIALES (Repaso)}
\\   Si en un espacio $R^{n}$ se define la suma de dos de sus elementos y el producto de un escalar por un vector, Entonces podemos hablar de un espacio vectorial.
\\   Si x = ($x_{1}$, $x_{2}$) e y = ($y_{1}$, $y_{2}$) $\epsilon$ $R^{2}$ $\Rightarrow$ (x + y) $\epsilon$ $R^{2}$ 
\\
\\
 Si x = ($x_{1}$, $x_{2}$) $\epsilon$ $R^{2}$ y $k$ es un escalar  $\Rightarrow$ $k$x  $\epsilon$ $R^{2}$ 

    \item ESPACION METRICO
Es un espacio vectorial donde se introduce una metrica o distancia y se simboliza  (\textit{ $R^{n}$, d})

    \item DISTANCIA
Sea el conjunto M distinto de vacio y  los elementos  x , y , z  pertenecientes a M, definimos la distancia como una funcion positiva, real , escalar que cumple las siguientes propiedades.

Propiedades de la distancia
\begin{enumerate}
\item La distancia a si mismo es nula.
\\ \textit d({x,y}) = 0 $\Longleftrightarrow$  x = y

\item Propiedad simetrica
\\ \textit d({x,y}) = \textit d({y,x})


\item Propiedad triangular.
\\ \textit d({x,z}) $\leq$ \textit d({x,y}) + \textit d({y,z}) \\Esta propiedad se importante ya que la igualdad solo se dar\'a cuando los tres puentos esten alineados sobre una recta, en cualquier otro caso la desigualdad se cunplir\'a.
\\EJEMPLOS DE DISTANCIAS :\\
Distancia pitagorica: d(x,y) = $\sqrt{(x_{1} - y_{1})^{2} + (x_{2} - y_{2})^{2}}$.

 Distancia taxy: es la distancia como si recorrieras las calles no podes ir en diagonal
 d(x,y) =  |$x_{2}$ – $x_{1}$| +| $y_{2}$ – $y_{1}$|.


Distancia Ajedres: Es como si te movieras como el caballo del ajedres.
d(x,y) =  max {  |$x_{2}$ – $x_{1}$| + | $y_{2}$ – $y_{1}$|}


\end{enumerate}



    \item Garbage colector: quita los objetos a los que no haga referencia nada
\end{itemize}

\end{document}