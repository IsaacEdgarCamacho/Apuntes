\documentclass[a4paper,12pt]{article}

%%%%%%%%%%%% PREÁMBULO %%%%%%%%%%%%%%%%%%%%%

% Paquetes

\usepackage[utf8]{inputenc}
\usepackage[spanish,es-tabla]{babel}
\usepackage[T1]{fontenc}

\usepackage{listings}

% Comandos

\renewcommand{\lstlistingname}{Código}
\renewcommand{\lstlistlistingname}{Índice de fragmentos de código fuente}

% Opciones

\title{7114 Modelos y Optimizaci\'on 1 fiuba}
\author{Isaac Edgar Camacho}

%%%%%%%%%%%%%%%%%%%%%%%%%%%%%%%%%%%%%%%%%%%%%%
%margenes

\setlength{\textwidth}{120mm}
\setlength{\textheight}{200mm}



\begin{document}
\maketitle

\begin{abstract}

Este es un modesto aporte para los alumnos de la f\'acultad de ingenier\'ia  de la UBA de las carreras de sistemas e inform\'atica, para una primer materia, sobre economia avanzada.
\\
El web site de la m\'ateria es wwww.ModelosUno.com.ar

\end{abstract}

\title{\textbf{Conceptos previos}}
\begin{itemize}

\item Eficiencia
\item eficacia
\item Costo de oportunidad
\item valor marginal

\end{itemize}


\title{\textbf{Trabajando con m\'odelos matem\'aticos lineales}}
\\
\\
\textbf{¿Para que hacer un modelo?}
\\
Todo modelo de la realidad es una porcion de la misma y debe tener las mismas caracteristicas.
\begin{itemize}
\item \textbf{Economia de recursos:} si no tengo escazes de recursos directamente no hago modelos.
\item \textbf{Eficiencia}: se debe lograr mas con menos.
\item \textbf{simplicidad}: puedo mediante abstracci\'on lograr un modelo mas sencillo y eliminar la complejidad inherente del problema.
\item \textbf{En resumen es mejor que hacer multiples ensayos.}
\end{itemize}


Los mod\'elos se aplican a problemas de desici\'on y este existe cuando existen formas alternativas de actuar, con distintos resultados y diferentes eficiencias para lograr el objetivo es decir existen dudas respecto del curso alternativo a utilizar.
\\
\textbf{Elementos de un modelo}
\\
\textbf{Hipótesis y supuestos:}Para simplificar el modelo se delimita el sistema en estudio a través de las hipótesis y
supuestos simplificativos. Así se comienza a transformar el sistema físico en un modelo simbólico.
Las hipótesis deben ser probadas científicamente. Los supuestos son hipótesis que no pueden probarse.
\\
\textbf{Objetivo: }Mide la eficiencia de nuestro sistema. Surge como respuesta a tres preguntas:
\\
¿Qué hacer?
\\
¿Cuándo? (período de tiempo)
\\
¿Para qué?
\\
\textbf{Actividad}
Proceso unitario que se realiza en el sistema físico caracterizado por consumir recursos
y/o generar un resultado económico y/o indicar un estado.
Variables
Son las que miden o indican el estado de una actividad.
Las que miden pueden ser continuas o enteras.
Las que indican son, generalmente, variables (0,1) o bivalentes



\end{document}