%%%% PROCESAR con PdfLaTeX !!!!!


\documentclass[12pt]{book}
\usepackage{geometry}\geometry{top=2cm,bottom=2cm,left=3cm,right=3cm}
\usepackage{amssymb}
\usepackage{amsmath}
\usepackage{graphicx}
\usepackage{txfonts}

\usepackage[hidelinks]{hyperref}
\usepackage[spanish]{babel}
\setcounter{tocdepth}{3}
\usepackage[usenames]{color}

\usepackage{framed}
\usepackage{color}
\usepackage{wrapfig}\definecolor{shadecolor}{RGB}{224,238,238}

\usepackage[hidelinks]{hyperref}
\hypersetup{
    colorlinks=true,
    linkcolor=blue,
    filecolor=magenta,      
    urlcolor=cyan,
    pdftitle={Sharelatex Example},
    bookmarks=true,
%    pdfpagemode=FullScreen,
}

\begin{document}
\thispagestyle{empty}

\begin {center}

\includegraphics[scale=.4]{Logo-fcenuba.png}

\medskip
UNIVERSIDAD DE BUENOS AIRES

Facultad de Ciencias Ex\'actas y Naturales

Departamento de Matem\'aticas


\vspace{3cm}


\textbf{\large ÁLGEBRA I}

\vspace{2cm}


Este es un modesto aporte para los alumnos de la f\'acultad de Ciencias Exactas y Naturales de la UBA de las carreras de licenciatura en Matemática y Computaci\'on.
De ninguna man\'era pretende ser una gu\'ia de estudio, ni remplaza las clases presenciales, el material oficial de la catedra esta disponible en el web site de la m\'ateria.
\\
http://cms.dm.uba.ar/

\end {center}


\vspace{2.5cm}

\noindent Autor:\,	Isaac Edgar Camacho Ocampo
 
\noindent Carrera:\,	Licenciatura en Ciencias de la Computaci\'on

\vspace{1cm}

\vspace{1cm}

\noindent Buenos Aires, 2020
\\ 
\\
\textit{Si se encuentra alg\'un error u omisi\'on en este res\'umen por favor colaborar en \\
\url{https://github.com/IsaacEdgarCamacho/Apuntes/tree/master/Algebra/} \quad \\ o escribirme a \\ \url{isaac.edgar.camacho@gmail.com}
}

\newpage


\tableofcontents

\tableofcontents
\chapter{Introducción}
\section{Conocimientos previos}
\section{Estado del arte}


\chapter{Conjuntos, Relaciones y Funciones.}
\section{Conjuntos.}
\subsection{Conjuntos y subconjuntos, pertenencia e inclusión.}
\subsubsection{Definición 1.1.1. (informal de conjunto y elementos.)}
Un conjunto es una colección de objetos, llamados elementos, que tiene la
propiedad que dado un objeto cualquiera, se puede decidir si ese objeto es
un elemento del conjunto o no.
\\
\\
\textit{Ejemplos:}
\begin{itemize}
\item $A = \{1, 2, 3\} ,\quad B = \{\triangle, \square, \lozenge \} , \quad  C = \{1, \{1\}, \{2, 3\}\} .$
\item $N = \{1, 2, 3, 4, \dots \} \quad $ el conjunto de los números naturales.
\end{itemize}


\subsection{Operaciones entre conjuntos}

\begin{shaded}
\textbf{Teorema }:si un conjunto A esta incluido en otro conjunto B, entonces el complemento de B esta incluido en el complemento de A
\end{shaded}
\begin{center}$ A\subseteq B \Rightarrow  B^{c} \subseteq A^{c}$\end{center}


\textit{Demostracion:}

\begin{equation*}
\begin{split}
Hitpotesis:   \quad  A \subseteq B \longleftrightarrow \quad \forall x \epsilon A \Rightarrow x \epsilon B  \\
&\equiv  \quad  def( a) \wedge def(i) \wedge \ ( 0 \leq i < \vert a \vert ) \wedge (0 \leq i+1 < \vert a \vert ) \quad      \wedge_{L} \ True  \\ \\
wp(S, \ Post) \ &\equiv  \quad  ( 0 \leq i < \vert a \vert ) \wedge (0 \leq i+1 < \vert a \vert )
\end{split}
\end{equation*}

\begin{equation*}
\begin{split}
wp(S, \ Post) \  &\equiv \quad wp ( \ result  := a[i] + a[i+1] \ , \ Post^{result}_{a[i] + a[i+1]} \ )  \\ \\
&\equiv  \quad  def( a[i] + a[i+1]) \quad \wedge_{L} \quad \ \underbrace{a[i] + a[i+1] = a[i] + a[i+1]}_{\text{es siempre True}}  \\
&\equiv  \quad  def( a) \wedge def(i) \wedge \ ( 0 \leq i < \vert a \vert ) \wedge (0 \leq i+1 < \vert a \vert ) \quad      \wedge_{L} \ True  \\ \\
wp(S, \ Post) \ &\equiv  \quad  ( 0 \leq i < \vert a \vert ) \wedge (0 \leq i+1 < \vert a \vert )
\end{split}
\end{equation*}

El Teorema de Pitágoras lleva este nombre porque su descubrimiento recae sobre la escuela pitagórica. (Fuente: Wikipedia)


\begin{shaded}
\textbf{Ejercicio: }
Probar usando el principio de inducción que 7 divide a $ 15^{n} + 7$ para todo $n \in \mathbb N $  
\end{shaded}

\begin{wrapfigure}[0]{r}{0.5\textwidth}
\vspace{-15pt}
\begin{shaded}
\textbf{Ejercicio: } Probar usando el principio de inducción que 7 divide a $ 15^{n} + 7$ para todo $n \in \mathbb N $  
\end{shaded}
\end{wrapfigure}

\begin{equation*}
\begin{split}
7\mid (15^{n} + 7) \  &\equiv \quad wp ( \ result  := a[i] + a[i+1] \ , \ Post^{result}_{a[i] + a[i+1]} \ )  \\ \\
&\equiv  \quad  def( a[i] + a[i+1]) \quad \wedge_{L} \quad \ \underbrace{a[i] + a[i+1] = a[i] + a[i+1]}_{\text{es siempre True}}  \\
&\equiv  \quad  def( a) \wedge def(i) \wedge \ ( 0 \leq i < \vert a \vert ) \wedge (0 \leq i+1 < \vert a \vert ) \quad      \wedge_{L} \ True  \\ \\
wp(S, \ Post) \ &\equiv  \quad  ( 0 \leq i < \vert a \vert ) \wedge (0 \leq i+1 < \vert a \vert )
\end{split}
\end{equation*}




\section{Ejercicios}


\begin{wrapfigure}[0]{r}{1\textwidth}
\vspace{-15pt}
\begin{shaded}
\textbf{Ejercicio: } Probar usando el principio de inducción que 7 divide a $ 15^{n} + 7$ para todo $n \in \mathbb N $  
\end{shaded}
\end{wrapfigure}


\chapter{Divisibilidad.}
Divisibilidad
Cuando hablamos de divisibilidad en el conjunto de los números enteros nos referimos a la división entera podemos señalar que tanto la suma el producto y la resta son operaciones cerradas dentro del conjunto de los enteros es decir como resultado obtenemos otro entero pero no ocurre esto en el caso de la división excepto para algunos casos 

Dados a y b enteros diremos que a divide a b si podemos escribir a a como múltiplo de b

\section{Cociente y resto}
Cuando somos chicos aprendemos que 6 “cabe” cuatro veces en 27 y el resto es 3, o sea
\[ 27 = 6 \cdot 4 + 3. \]
Un punto importante es que el resto debe ser menor que 6. Aunque, también es verdadero que,
por ejemplo
\[27 = 6 \cdot 3 + 9\]
debemos tomar el menor valor para el resto, de forma que “lo que queda” sea la más chico posible.

\subsection{Suposiciones}
\subsection{Modelos}
\section{Resultados preliminares}
\section{Resultados postprocesados}
\subsection{Valores atípicos}
\subsection{Correlaciones}
\chapter{Conclusiones}





\begin{shaded}
\textbf{Ejercicio:} Probar usando el principio de inducción que 7 divide a $ 15^{n} + 6$ para todo $n \in \mathbb N_{0}$  
\end{shaded}

Cuando trabajamos con inducci\'on debemos definir cual es la afirmacion que vale para cada n   y no se pone para todo n, porque eso es lo que queremos probar.\\
\[
 P(n): 7 \mid 15^{n} + 6
\]
\newline
\textbf{Caso base $P(0): n = 0  \ $}\quad Queremos ver si el primer numero del conjunto cumpler P(n)

\[
(n=0 ) \ \Rightarrow \ 7 \ divide \ a \ (15^{0} + 6 )
\]

\[  15^{0} + 6 = 7  , \  7 = 1 \cdot 7  \ \Rightarrow 7\mid 7 \]
\\
Por lo tanto la afirmaci\'on para $P(0)$  es verdadera ya que 7 divide a 7.
\newline
\newline
\textbf{Paso inductivo} Suponemos que $P(n)$ es verdadera y probamos la veracidad de $P(n+1)$
\[
\text {Hip\'otesis  Induct\'iva } \ (HI) \equiv \ (15^{n} + 6 = k \cdot 7) ,\ con \ k \in \mathbb{Z} 
\]
\newline
Lo que queremos hacer es que suponiendo que $P(n)$ es verdad probamos la verdad de $P(n+1)$

\begin{equation*}
\begin{split}
15^{n+1} + 6 \quad   &\equiv \quad  15^{n+1} + 6   \\
&\equiv  \quad   15^{n} \cdot 15 + 6  \\
&\equiv  \quad   15^{n} \cdot (1 + 14) + 6 \\
&\equiv  \quad   15^{n} \cdot 1 + 15^{n} \cdot 14 + 6 \\ 
&\equiv  \quad   (15^{n} \cdot 1 + 6) +15^{n} \cdot 14  \\
HI \quad   &\equiv  \quad   k \cdot 7 +15^{n} \cdot 14  \\
\quad   &\equiv  \quad   k \cdot 7 +15^{n} \cdot (2 \cdot 7)  \\
15^{n+1} + 6 \quad   &\equiv  \quad  7 \cdot \underbrace{ (k +15^{n} \cdot 2)}_{ esto \ \in \mathbb{Z}}  \\
\end{split}
\end{equation*}
Podemos ver que $ 15^{n+1} + 6  $ se expresa como multiplo de 7, por lo tanto queda demostrado nuestro ejercicio

\begin{shaded}
\textbf{Ejercicio:} Probar usando el principio de inducción que 10 divide a $ 16^{n} - 6^{n}$ \\ para todo $n \in \mathbb N$  \\
\[	
	10  \mid 16^{n} - 6^{n} \quad \forall n \in \mathbb{Z}
\]
\end{shaded}

\begin{equation*}
\begin{split}
\textbf{Caso Base P(1) } & \quad n=1 \\
16^{1} - 6^{1} \quad   &\equiv \quad  16 - 6 = 10 \\
&\equiv \quad  10 = 1 \cdot 10 \text{\quad (con esto vemos que cumple)}\\ \\
\textbf{Paso inductivo P(n)} & \quad P(n) \text { es verdad y pruebo para } P(n+1)	\\
\text{Hip\'otesis inductiva } &\Rightarrow \quad (16^{n} - 6^{n} = k \cdot 10) \text{ con } k \in \mathbb{Z}\\ \\
16^{n+1} - 6^{n+1} \quad &\equiv  \quad   16^{n} \cdot 16 - 6^{n} \cdot 6  \\
&\equiv  \quad   16^{n} \cdot (6+10) - 6^{n} \cdot 6  \\
&\equiv  \quad   (16^{n} \cdot 6 + 16^{n} \cdot10) - 6^{n} \cdot 6  \\
&\equiv  \quad    16^{n} \cdot 6 - 6^{n} \cdot 6  +  16^{n} \cdot 10 \\
&\equiv  \quad   6 \cdot \underbrace{ (16^{n} - 6^{n}) }_{HI}  +  16^{n} \cdot 10 \\
HI \quad   &\equiv  \quad  6 \cdot k \cdot 10 + 16^{n} \cdot 10 \\
16^{n+1} - 6^{n+1} \quad  &\equiv  \quad  10 \cdot (\underbrace{  k \cdot 6 + 16^{n})}_{ esto \ \in \mathbb{Z}}  
\end{split}
\end{equation*}
Podemos ver que $ 16^{n+1} - 6^{n+1} $ se expresa como multiplo de 10, por lo tanto queda demostrado nuestro ejercicio


\chapter{Congruencia}

De manera informal dos números $a,b \in \mathbb{Z}$ son congruentes cuando tienen el mismo resto al dividirlo por un tercero m que vamos a llamar módulo.
\begin{equation*}
a \equiv b (m) \quad \Leftrightarrow \quad  r_m(a) = r_m(b) \quad  \Leftrightarrow \quad  m \vert (a-b)
\end{equation*}

es equivalente decir que la resta de esos dos números que llamamos congruentes es un múltiplo de ese tercer número que llamamos módulo



\end{document}
